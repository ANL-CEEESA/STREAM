% File: report.tex
% Author: David Thierry
% 
% vim: tw=80 fo=cqt
% uploaded Nov 30 11:14 am
\documentclass{amsart}

\usepackage[english]{babel}
\usepackage{amsmath}
\usepackage{bm}


\usepackage[letterpaper, margin=1in]{geometry}

\usepackage{float}
\usepackage{tabularx}
\usepackage{longtable}
\usepackage{hyperref}
\usepackage{graphicx}
\usepackage[title]{appendix}

%
\DeclareFontFamily{U}{matha}{\hyphenchar\font45}
\DeclareFontShape{U}{matha}{m}{n}{ <-6> matha5 <6-7> matha6 <7-8>
matha7 <8-9> matha8 <9-10> matha9 <10-12> matha10 <12-> matha12 }{}
\DeclareSymbolFont{matha}{U}{matha}{m}{n}
\DeclareMathSymbol{\myvee}{\mathbin}{matha}{"5F}
\newcommand{\myveebar}{\mathbin{\vcenter{\hbox{$ \underline{\mkern-2.4mu{\myvee}\mkern-2.4mu} $}}}}


\begin{document}

\title{\texttt{DRE4M} discrete version equations.}
\author{David Thierry \{dthierry@anl.gov\}}


\date{\today}
\maketitle
%\pgfdeclarelayer{background}
%\pgfdeclarelayer{foreground}
%\pgfsetlayers{background,main,foreground}

\begin{abstract}
%------------------------------------------------------------------------------
The diversification of the technological makeup of all economic sectors towards
higher efficiency, carbon neutrality, and cost-effective technologies, is a key
challenge towards achieving carbon neutrality and
stabilizing global warming by 2050.  
In order to achieve the projected climate goals, it is required to generate plans
that restructure the different sectors across the economy to meet the target
greenhouse gas (GHG) output levels, 
while also considering the technological, economic, and 
social issues associated with it.
This is a nontrivial task because these economic sectors have substantial 
heterogeneity and complex interactions among its components, therefore an 
analysis reflecting the planning of feasible technological changes toward 
net neutrality, needs adaptability to varying degrees of granularity.

In the face of this situation, this manuscript proposes a mathematical 
framework to address the strategies in
the technological diversification of a sector through retirement,
and retrofitting of its existing assets or capacity, alongside with the
deployment of new capacity from a set of available technologies. This is
simultaneously done with the 
accounting for carbon dioxide emissions, energy requirements, and demand
constraints, while minimizing the Net Present Value (NPV) of the system over a
time horizon.  The proposed framework is denominated as the Decarbonization Road
mapping and Energy, Environmental, Economic, and Equity Analysis Model (DRE$^4$M).
It was implemented in the Julia language intended to be a piece of open-source
software.

\end{abstract}
%\listoftodos
%------------------------------------------------------------------------------
\section{Methodology}\label{methodology}
%
\subsection{Sets, initial assumptions, and notation}
%
Throughout this section variables and constraints are defined with respect to
the following sets, 

\begin{tabular}{@{}ll@{}}

    $I$	&	Technology set ($\left\lbrace \text{tech 0}, \text{tech 1}, 
    \dots\right\rbrace$)	\\
    $I_F$	&	Fuel-based technology set ($I_F \subseteq I$) \\
    $I_C$	&	Carbon-based technology set ($I_C \subseteq I_F$) \\

    $K_i$	&	Sub-technology for existing assets	\\
    $\tilde{K}_i$	&	Sub-technology for new assets	\\
    $L_i$	&	Above service life set for existing assets 
    ($L_i \subseteq T \times N_i$)	\\
    $\tilde{L}_i$	&	Above service life set for new assets ($\subseteq T 
    \times T$).	\\
    $N_j$	&	Initial age of existing asset set ($N_j \subseteq 
    \mathbb{Z}_{\geq 0}$)	\\    
    $T$	&	Time set ($\left\lbrace 0, 1, \dots, \mathtt{horizon}
    \right\rbrace$)	\\
\end{tabular}
%

\smallskip

All the variables presented in the following subsections are indexed by the
time, technology index, and a proxy of age, unless specified otherwise. 
Three kinds of assets are
recognized within the framework – existing, retrofitted, and new. The term
`asset' is used in broad sense, depending of the context, although in every 
instance, an asset is something takes in energy, material, labor, or economic 
inputs and uses that to make economically useful energy or material products and wastes.

An asset can be a \emph{production unit}, e.g., a coal-fired power plant, or it 
can refer to \emph{production capacity}, e.g., the \emph{ton/hr} of cement produced. 
An `existing asset' refers to an asset that was in place prior to the start of the 
analysis time horizon.
%
For existing and retrofitted assets of kind $i\in I$, the set $N_i$ represents
the age distribution at the beginning of the horizon, e.g.\ $N_{\text{tech 0}}
:= \left\lbrace 2, 4\right\rbrace$ implies that at $t=0$ the assets of kind
`tech 0' have ages of `2' and `4'. New assets, on the other hand, use the
time set $T$ for their initial age because all the new allocations happen at
some point within the horizon. The age of an asset in any given time step can be
inferred using the time index and the initial age of the asset in accordance with 
the kind of asset (new, retrofitted, existing). For existing and retrofitted assets, 
age can be calculated as $\mathtt{age} = t+j$, and for new assets, $\mathtt{age} = t-j,\; t\geq j$.
%

The time, technology, and initial age of a variable is shown using a
combination of superscripts and subscripts, e.g., $x^t_{i,k,j}$, where $t$
denotes time, and the subscripts $(i,k,j)$ denote technology, sub-technology
and initial age respectively. Both retrofitted and new assets possess a
sub-technology set $K_i$ and $\tilde{K}_i$ respectively, which indicates that
these assets derive from a base technology $i\in I$, e.g.\ a coal-fired power
plant with carbon capture derives from a base coal-fired plant.

% Other sets used in % the model, include $\tilde{K}_i$ which denotes the sub-technology 
%set for % \emph{new} assets, $L_i$ and $\tilde{L}_i$, which are both subset of the
% product of the time and age sets for existing and new assets respectively.  
%All
%the sets mentioned thus far are summarized in Table~\ref{tab:sets}.
%
The variables of the model are assumed to be non-negative, unless specified
otherwise. The current model has been devised to be linear. However, the nature 
of the variables is contingent upon the context, for instance,  
modelling a small group of residential buildings versus modelling 
the aggregate electric power generation capacity of the United States. 
%As a consequence, the computational tractability
%will be tied to the case study.
%i
The variables representing building assets denote entities of indivisible 
nature, i.e., integer variables representing an individual building. 
On the other hand, the electricity generation assets could be assumed 
to represent \emph{capacity} in a continuous manner, i.e., continuous 
variables representing MW of electric generation capacity. It follows 
that the core methodology of this manuscript applies to both continuous 
and discrete variables, and that the nature of these variables will 
require solution strategies of significantly different computational 
complexities. For the purposes of demonstration of the problem framework 
and discussion of the results of the case study that follow, we will 
assume a continuous treatment of asset variables.
Next is a list of the variables and parameters with their descriptions
used in the next subsection.


\begin{centering}
\begin{tabular}{@{}ll@{}}
    \textbf{Variables:} &   \\ 
    $\mathtt{NPV}$          & net present value      \\
    $\mathtt{softSl}$          &   overall soft service life penalty    \\
    $\mathtt{termCost}$          & overall terminal cost      \\
    %
    $u^t_{i,j}$	&	retired assets	amount\\
    $\overline{u}^t_{i,k,j}$	&	retired retrofitted assets amount.	\\
    %
    $v^t_{i,k,j}$	&	retired new assets amount	\\
    %
    $w^t_{i,j}$	&	existing assets	amount \\
    $W^t_{i,j}$	&	effective capacity	\\
    $\mathtt{wCo}^t_{i,j}$	&	carbon dioxide emission for existing assets	\\
    $\mathtt{wFuelC}^t_{i}$	&	fuel cost for existing assets	\\
    $\mathtt{wFixOnM}^t_i$	&	fixed cost of O\&M for existing assets	\\
    $\mathtt{wHeat}^t_{i,j}$	&	heat requirement for existing	\\
    $\mathtt{wProd}^t_{i,j}$	&	production of existing assets	\\
    $\mathtt{wRetC}^t_{i,j}$	&	retirement cost	\\
    $\mathtt{wSoftSl}$	&	soft retirement penalty	\\
    $\mathtt{wTermC}_{i,j}$	&	terminal cost	\\
    $\mathtt{wVarOnM}^t_{i}$	&	variable cost of O\&M for existing assets\\
    %
    $x^t_{i,k,j}$	&	new assets amount	\\
    $X^t_{i,k,j}$	&	effective new capacity	\\
    $\tilde{x}^t_{i,k}$	&	new asset allocation	\\
    $\mathtt{xCap}^t_{i,k}$	&	overnight capital cost for new assets	\\
    $\mathtt{xProd}^t_{i,k,j}$	&	production of new assets	\\
    $\mathtt{xRetC}^t_{i,k,j}$	&	retirement cost for new assets	\\
    $\mathtt{xSoftSl}$	&	soft retirement penalty for new assets	\\
    $\mathtt{xTermC}_{i,k,j}$	&	terminal cost for new assets	\\
    %
    $y^t_{i, k,j}$	&	transitioned to retrofitted assets amount	\\
    %
    $z^{t}_{i,k,j}$	&	retrofitted assets amount	\\
    $Z^t_{i,k,j}$	&	effective retrofit capacity	\\
    $\mathtt{zCap}^t_{i,k}$	&	overnight capital cost for retrofitted assets\\
    $\mathtt{zRetC}^t_{i,k,j}$	&	retirement cost for retrofitted assets\\
    $\mathtt{zSoftSl}$	&	soft retirement penalty for retrofitted assets\\
    $\mathtt{zTermC}_{i,k,j}$	&	terminal cost for retrofitted assets\\
\end{tabular}
\end{centering}

\begin{centering}
    \begin{tabular}{@{}ll@{}}
    \textbf{Parameters:} &   \\
    $\mathtt{co2Budget}^t$	&	carbon dioxide budget	\\
    $\mathtt{cFact}_{i,j}$	&	capacity factor	\\
    $\Delta_{i,k}$              &       Lead time for new assets\\
    $\overline{\Delta}_{k,j}$   &       Lead time for retrofitted assets.\\
    $\mathtt{demand}^t$	&	effective capacity	\\
    $\mathtt{initCap}_{i,j}$    &       Initial asset/capacity matrix\\
    $\mathtt{kTime}$	&   scaling factor for time		\\
    $\mathtt{kTermC}$	& scaling scalar for terminal cost		\\
    $\mathtt{fFuelC}_{i,k}(t,j)$	&   (function) fuel cost per heat unit	\\
    $\mathtt{fFixCost}_{i,k}(t,j)$	&   (function) O\&M fixed cost	\\
    $\mathtt{fHeatRate}_{i,k}(t,j)$	&   (function) heat rate per production
    unit \\
    $\mathtt{fVarCost}_{i,k}(t,j)$	&   (function) O\&M variable cost	\\
    $\mathtt{fPenalty}_{i}$	&	(function) service life penalty	\\
    $\mathtt{fRetCost}_{i,k}$	&	(function) retirement cost	\\
    $\mathtt{sLife}_i$	&   service life \\
    \end{tabular}
\end{centering}


\subsection{Constraints}
%%%%%%%%%%%
\subsubsection{Core: Asset tracking}
%
\noindent 
%
%------------------------------------------------------------------------------
At the core of the model is a \emph{stock-and-flow engine}, with equations 
containing decision variables that govern and track the evolution of assets 
over time. The variables used in this section represent three possible states 
of an asset – existing, retirement, and allocation. 
%Next is a list of the variables and 
%parameters followed by the model constraints and their descriptions.

%\smallskip
%\noindent
%\textbf{Variables:} \\
%\begin{tabular}{@{}ll@{}}
%    \textbf{Variables:} &   \\ 
%    $v^t_{i,k,j}$	&	retired new assets amount	\\
%    $w^t_{i,j}$	&	existing assets	amount \\
%    $u^t_{i,j}$	&	retired assets	amount\\
%    $\overline{u}^t_{i,k,j}$	&	retired retrofitted assets amount.	\\
%    $x^t_{i,k,j}$	&	new assets amount	\\
%    $\tilde{x}^t_{i,k}$	&	new asset allocation	\\
%    $y^t_{i, k,j}$	&	transitioned to retrofitted assets amount	\\
%    $z^{t}_{i,k,j}$	&	retrofitted assets amount	\\
%\end{tabular}


%\smallskip
%\noindent
%\textbf{Parameters:} \\
%\begin{tabular}{@{}ll@{}}
%    \textbf{Parameters:} &   \\ 
%    $\mathtt{initCap}_{i,j}$    &       Initial asset/capacity matrix\\
%    $\Delta_{i,k}$              &       Lead time for new assets\\
%    $\overline{\Delta}_{k,j}$   &       Lead time for retrofitted assets.\\
%\end{tabular}
%Let $w^t_{i,j}$ represent an asset of kind $i \in I$ of \emph{initial} age $j
%\in N_i$ at some period $t\in T$. 
\paragraph{Existing asset balance.}
%
In the model it assumed that any existing asset or capacity has two possible
pathways in time, i.e.\ retrofitting and retirement. Let $w^t_{i,j},
y^t_{i,k,j},u^{t}_{i,j}\in\mathbb{R}_{\geq 0}$ denote the existing, transition
to retrofitted and retired assets respectively. Then, at time $t$, an existing
asset can remain the same, transition to retrofitted asset, or get retired. This
dynamic is shown in Figure~\ref{fig:wuyyy}, and it can be interpreted as the
existing asset balance equation as follows,
\begin{figure}[ht]
    \centering
    \includegraphics[width=0.5\linewidth]{./pics/wuyyy.pdf}
    \caption{Asset retirement and retrofitting.\label{fig:wuyyy}}
\end{figure}
%
\begin{equation}
  w^{t+1}_{i,j} = w^{t}_{i,j} - u^{t}_{i,j} - \sum_{k \in K_i} y^{t}_{i, k,j}, 
  \; \forall t \in T, i \in I, j \in N_i.
\label{eq:wbal}
\end{equation}
%
Once an asset is retired, it no longer plays any role in the energy system. 
%
\paragraph{Retrofit asset balance.}
%
The transition to retrofitted assets $y^t_{i,k,j}$ enter a separate balance
similar to the one of existing assets of Equation~\eqref{eq:wbal}. For this,
$z^t_{i,k,j}, \overline{u}^t_{i,k,j}\in\mathbb{R}_{\geq 0}$ represent the
existing retrofitted and retrofitted-retired assets. Their balance can be
represented as in Figure~\ref{fig:zuyy}, and is written as follows,
%
\begin{figure}[ht]
   \centering
   \includegraphics[width=0.5\linewidth]{./pics/zuyyy.pdf} 
   \caption{Retrofitted asset balance.} 
\label{fig:zuyy} 
\end{figure}
%
\begin{equation}
    z^{t+1}_{i,k,j} = z^{t}_{i,k,j} - \overline{u}^t_{i,k,j} + y^t_{i,k,j},\;
    \forall 
  t \in T, i \in I, k \in K_i, j \in N_i.
\label{eq:zeqzrf}
\end{equation}
%
%The retrofitted assets modify their base counterparts properties in a way that
%may be favorable to the system. 
%The retrofits provide flexibility to the energy system by being available at a
%relatively low-cost while overriding some of the properties of the underlying
%assets. 
%
\paragraph{New asset balance.}
As previously described, existing assets can experience a number of outcomes
over time. However, additional assets on top of the existing ones might be
required, for instance, if demand grows, or assets of a certain kind become
prohibited from take part in the system.  The distinction between \emph{new}
and \emph{existing} assets is made as whether the asset was created at some
point before time $t=0$ (existing) or after (new). Moreover, new assets do not
necessarily have the same properties of the existing counterparts, e.g.\ because
of improvements in the state of art of technology.  It is noted that, though the
retrofit of a new is conceptually similar to the ideas depicted in
Figure~\ref{fig:wuyyy}, these are not enabled for the current iteration of the
model.  Therefore, let $x^{t}_{i,k,j}, v^t_{i,k,j}\in \mathbb{R}_{\geq 0}$
represent the new and retired-new assets respectively, and the balance for new
assets is shown in Figure~\ref{fig:xux} and written as follows,
%
\begin{figure}[ht]
   \centering
   \includegraphics[width=0.5\linewidth]{./pics/xux2.pdf} 
   \caption{New asset balance.} \label{fig:xux} 
\end{figure}
%
\begin{equation}
    x^{t+1}_{i,k,j} = x^{t}_{i,k,j} - v^t_{i,k,j} 
    ,\; \forall 
    t,j \in T, i \in I, k \in \tilde{K}_i.
\label{eq:xbal}
\end{equation}
%
\paragraph{New asset allocation.}
The dynamics of new assets are predicated by the allocation decision, i.e.\ the
amount of new assets created at a particular time. These are represented
by $\tilde{x}^t_{i,k}\in\mathbb{R}_+$ and it is linked to the new assets through
the next equation,
%
\begin{equation}
    x^t_{i,k,t} = \tilde{x}^t_{i,k}, \; 
    \forall t\in T, i\in I, k\in \tilde{K}_i.
    \label{eq:xeqxal}
\end{equation}
%
\paragraph{Initial capacity.}
%
The existing assets at time $t=0$ is exogenous information which is linked to
the model with the following equation, 
%
\begin{equation}
    w^0_{i,j} = \mathtt{initCap}_{i,j}, \; \forall i\in I, j\in N_i,
\end{equation}
%
in which the matrix $\mathtt{initCap}_{i,j}$ has the data indexed by technology
and age. Initial retrofitted capacities ($z^0_{i,k,j}$) are assumed to be zero,
i.e.,
%
\begin{equation}
    z^0_{i,k,j} = 0, \; \forall i\in I, k \in K_i, j\in N_i.
\end{equation}
%
%
%%%%%%%%%%%
\paragraph{Lead times modifications.}
%%%%%%%%%%%
To account for the delays that occur from the allocation decision to the point
of deployment, modifications on constraints~\eqref{eq:xeqxal} 
and~\eqref{eq:zeqzrf} are required. 
%
%One of the issues with the deployment of new capacity and retrofitting is the
%time amount that exists between the decision making and the beginning of
%operation of the capacity. 
%
%These time constraints might be associated with the construction, connection
%with supply, etc. 
%
%The timing considerations impact the
%decisions of the kind of technology/retrofit that can be deployed, e.g.\ if a
%cost-effective decision can not be available in time to satisfy demand. 
Let $\Delta_{i,k}$ denote the lead time for new allocations of kind $i,k$, then
every new allocation is followed by an offline period given by $\Delta_{i,k}$ as
shown in Figure~\ref{fig:leadTime}. This condition is reflected by modifying
Equation~\eqref{eq:xeqxal} as follows,
%
\begin{figure}[ht]
    \centering
    \includegraphics[width=0.5\linewidth]{./pics/leadtime.pdf}
    \caption{Lead time: there is a $\Delta_{i,k}$ time amount for which new
    capacities will be offline after allocation.} \label{fig:leadTime}
\end{figure}
%
\begin{equation}
    x^t_{i,k,t} = \tilde{x}^{t-\Delta_{i,k}}_{i,k}, \; 
    \forall t,j\in T, i\in I, k\in \tilde{K}_i,
\end{equation}
%
in which the right-hand-side contains the lead time effect. 
%
Similarly, the effect on the retrofit lead time $\overline{\Delta}_{k,j}$
modifies Equation~\eqref{eq:zeqzrf}, i.e.
%
\begin{equation}
    z^{t+1}_{i,k,j} = z^{t}_{i,k,j} - \overline{u}^t_{i,k,j} + 
    y^{t-\overline{\Delta}_{i,k}}_{i,k,j},\;
    \forall 
  t \in T, i \in I, k \in K_i, j \in N_i.
\end{equation}
%
%

\subsubsection{Core: Soft service life, retirement, and terminal cost}
%%%%%%%%%%%
%
By construction, the current iteration of the model allows assets to take part
in the system for an indefinite amount of time, in other words, a `soft'
approach for the service life is taken. As an asset moves forward into the
horizon some of its properties might be affected negatively. Moreover, assets
above the service life can be set to experience effects that bias
the solution toward rejecting them from the system, for instance by introducing
a penalty into the objective function. The advantage of soft
retirement is that assets of particular technologies can be set to have either
neutral or negative attitude toward above service life operation. 
%The following
%variable, parameters, and equations provide this capability to the model.
%

%\smallskip
%\noindent
%\textbf{Variables:} \\
%\begin{centering}
%    \begin{tabular}{@{}ll@{}}
%    \textbf{Variables:} &   \\ 
%        $\mathtt{wRetC}^t_{i,j}$	&	retirement cost	\\
%        $\mathtt{wSoftSl}$	&	soft retirement penalty	\\
%        $\mathtt{wTermC}_{i,j}$	&	terminal cost	\\
        %
%        $\mathtt{xRetC}^t_{i,k,j}$	&	retirement cost for new assets	\\
%        $\mathtt{xSoftSl}$	&	soft retirement penalty for new assets	\\
%        $\mathtt{xTermC}_{i,k,j}$	&	terminal cost for new assets	\\
        %
%        $\mathtt{zRetC}^t_{i,k,j}$	&	retirement cost for retrofitted 
%        assets\\
%        $\mathtt{zSoftSl}$	&	soft retirement penalty for retrofitted 
%        assets\\        
%        $\mathtt{zTermC}_{i,k,j}$	&	terminal cost for retrofitted 
%        assets	\\
%    \end{tabular}
%\end{centering}

%\smallskip
%\noindent
%\textbf{Parameters:} \\
%\begin{centering}
%    \begin{tabular}{@{}ll@{}}
%    \textbf{Parameters:} &   \\ 
%        $\mathtt{fPenalty}_{i}$	&	(function) service life penalty	\\
%        $\mathtt{fRetCost}_{i,k}$	&	(function) retirement cost	\\
%        $\mathtt{kTermC}$	& scaling scalar for terminal cost		\\
%        $\mathtt{sLife}_i$	&   service life \\
%    \end{tabular}
%\end{centering}
% The age of assets is generally reflected negatively in the associated
% quantities of the system, for instance the heat requirement might increase.
% Nevertheless, overly-aged assets or capacity must be taken offline in a
% meaningful way by the model. 
% For this purpose, a model can either have hard (i.e.\ assets are dropped at
% the service life) or soft (i.e.\ assets may continue after the service life)
% approach for retirement.
% This implies that it is possible to have assets above the service life, however
% these sustain an increasingly higher form of penalty for every year above it.
% For this purpose, an arbitrary function for the penalty is required. Such
% function must increase monotonically with age, so that the older the capacity,
% the more likely is forced to be retired.
%
%
\paragraph{Soft service life penalty.}
Let the sets 
$L_i:=\{\left(t,j\right)\in T\times N_i:t+j\geq \mathtt{sLife}_i\}$
and $\tilde{L}_i:=\{\left(t,j\right)\in T\times T:t-j\geq \mathtt{sLife}_i\}$
contain the ages above service life for existing and new assets respectively,
i.e.\ a combination of time ($t$) and initial ages ($j$). Then, the soft service
life retirement terms for existing, retrofitted, and new assets are written as
follows respectively,
%
% \footnote{It must be noted that the equations
% and terms for retrofits can be easily inferred, therefore they are not shown
% throughout this section for brevity.} 
%
%
\begin{equation}
    \mathtt{wSoftSl} = \sum_{i\in I} \sum_{\left(t,j\right)\in L_i} 
    \mathtt{fPenalty}_i \left(t+j\right) \, w^t_{i,j},
    %\; \forall t \in T, i \in I, j \in N_i.
\end{equation}
%
\begin{equation}
    \mathtt{zSoftSl} = \sum_{i\in I} 
    \sum_{k \in K_i} \sum_{\left(t,j\right)\in L} 
    \mathtt{fPenalty}_i \left(t+j\right) \, z^t_{i,k,j},
    %\; \forall t \in T, i \in I, j \in N_i.
\end{equation}
%
and,
%
\begin{equation}
    \mathtt{xSoftSl} = \sum_{i\in I} 
    \sum_{k \in \tilde{K}_i} \sum_{\left(t,j\right)\in\tilde{L}} 
    \mathtt{fPenalty}_i \left(t-j\right) \, x^t_{i,k,j},
    %\; \forall t \in T, i \in I, j \in N_i.
\end{equation}
%
where $\mathtt{fPenalty}_i(\mathtt{age})$ is an arbitrary penalty function of
the age, whose value determines the bias toward letting the underlying asset
remain in the system. Typically, such function should monotonically increase with
age make older assets consistently less desirable. An example of the function
used is shown in Appendix~\ref{appendix}.
As referred previously, it is possible that only a subset of the technology
assets have soft retirement effects, therefore the first summation term is
enabled for such subset, while the remaining assets might remain in the system
indefinitely. 
%
\paragraph{Retirement cost.}
%
Regardless of the service life, assets might undergo retirement regularly as
part of the changes of the system over time, these are captured by the variables
$u^t_{i,j}, \overline{u}^t_{i,k,j}$, and $v^t_{i,k,j}$. Changes in the nature of
the asset, e.g.\ a retirement, have a cost component. These costs primarily
depend on the age of the asset, though additional special considerations can
also be allowed. Nevertheless, let the function $\mathtt{fRetCost}_{i,k}\!
\left(t,j\right)$ return the cost of retirement per asset unit as a function of
time and initial age, then the retirement costs are given as follows,
%
\begin{equation}
    \mathtt{wRetC}^t_{i,j} = \mathtt{fRetCost}_{i,0}\! \left(t,j\right)\, 
    u^t_{i,j},\; \forall t \in T, i \in I, j \in N_i,
\end{equation}
%
\begin{equation}
    \mathtt{zRetC}^t_{i,k,j} = \mathtt{fRetCost}_{i,k}\! \left(t,j\right)\,
    \overline{u}^t_{i,k,j},\; \forall t \in T, i \in I, k \in K_i
    j \in N_i,
\end{equation}
%
and,
%
\begin{equation}
    \mathtt{xRetC}^t_{i,k,j} = \mathtt{fRetCost}_{i,k}\! \left(t,j\right)\,
    v^t_{i,k,j}
    ,\; \forall 
    t,j \in T, i \in I, k \in \tilde{K}_i.
\end{equation}
%
Further details about the retirement cost per asset unit are given in the
Appendix~\ref{appendix}.
%
\paragraph{Terminal cost.}
%
Because of the finite planning horizon and the allocation scheme of the model,
new allocations toward the end of the horizon might be positively biased. These
primarily do not consider the costs generated by the assets found past the end
of the horizon. 
%
One way to deter this bias is by imposing a penalty on late allocations, i.e.\ a
terminal cost. This term can be conceptualized as an approximation of the costs
generated by the asset in an infinite horizon setting. Though a precise
approximation of these values might be possible, in the current framework using
a penalty proportional to the retirement cost at the latest period can be
assumed as an upper-bound. This are given as follows,
%is by relating it the costs associated
%with the assets at the end of the finite horizon plan in the infinite horizon
%setting. I.e., the look ahead costs that are not accounted in the current
%t horizon. 
%
\begin{equation}
    \mathtt{wTermC}_{i,j} = \mathtt{kTermC} \cdot
    \mathtt{fRetCost}_{i,k}\! \left(|T|,j\right)\, 
    w^{|T|+1}_{i,j}
    ,\; \forall i \in I, j \in N_i,
\end{equation}
%
\begin{equation}
    \mathtt{zTermC}_{i,k,j} = \mathtt{kTermC} \cdot
    \mathtt{fRetCost}_{i,k}\! \left(|T|,j\right)\, 
    z^{|T|+1}_{i,k,j},\; \forall
    j \in T, i \in I, k \in K_i,
\end{equation}
%
and,
%
\begin{equation}
    \mathtt{xTermC}_{i,k,j} = \mathtt{kTermC} \cdot
    \mathtt{fRetCost}_{i,k}\! \left(|T|,j\right)\, 
    x^{|T|+1}_{i,k,j},\; \forall
    j \in T, i \in I, k \in \tilde{K}_i,
\end{equation}
%
where $\mathtt{kTermC} \in \mathbb{R}_{\geq 0}$ is a scalar that must be set to
a large value such that the terminal cost is approximated.
%
% we need to say something about the base load capacity

\subsubsection{Superstructure: Asset associated constraints}
%%%%%%%%%%%
The remaining terms of interest pertain to quantities 
that derive from assets that participate in the system. 
For instance, the carbon dioxide generated, operation and
maintenance cost, etc. In order to construct such superstructure of terms,
auxiliary variables will be presented first, followed by the quantities of
interest, and finalized by special constraints required by the system. 

%\smallskip
%\noindent
%\textbf{Variables:} \\
%\begin{tabular}{@{}ll@{}}
%    \textbf{Variables:} &   \\ 
%    $W^t_{i,j}$	&	effective capacity	\\
%    $Z^t_{i,k,j}$	&	effective retrofit capacity	\\
%    $X^t_{i,k,j}$	&	effective new capacity	\\
%    $\mathtt{wProd}^t_{i,j}$	&	production of existing assets	\\
%    $\mathtt{xProd}^t_{i,k,j}$	&	production of new assets	\\
%    $\mathtt{wHeat}^t_{i,j}$	&	heat requirement for existing	\\
%    $\mathtt{wCo}^t_{i,j}$	&	carbon dioxide emission for existing assets	\\
%    $\mathtt{wFuelC}^t_{i}$	&	fuel cost for existing assets	\\
%    $\mathtt{wFixOnM}^t_i$	&	fixed cost of O\&M for existing assets	\\
%    $\mathtt{xCap}^t_{i,k}$	&	overnight capital cost for new assets	\\
%    $\mathtt{zCap}^t_{i,k}$	&	overnight capital cost for retrofitted assets\\
%    $\mathtt{wVarOnM}^t_{i}$	&	variable cost of O\&M for existing assets	\\
%    $\mathtt{NPV}$          & net present value      \\
%    $\mathtt{termCost}$          & overall terminal cost      \\
%    $\mathtt{softSl}$          &   overall soft service life penalty    \\
%\end{tabular}

%\smallskip
%\noindent
%\textbf{Parameters:} \\
%\begin{tabular}{@{}ll@{}}
%    \textbf{Parameters:} &   \\ 
%    $\mathtt{demand}^t$	&	effective capacity	\\
%    $\mathtt{co2Budget}^t$	&	carbon dioxide budget	\\
%    $\mathtt{cFact}_{i,j}$	&	capacity factor	\\
%    $\mathtt{kTime}$	&   scaling factor for time		\\
%    $\mathtt{fHeatRate}_{i,k}(t,j)$	&   (function) heat rate per production 
%    unit \\
%    $\mathtt{fFuelC}_{i,k}(t,j)$	&   (function) fuel cost per heat unit	\\
%    $\mathtt{fFixCost}_{i,k}(t,j)$	&   (function) O\&M fixed cost	\\
%    $\mathtt{fVarCost}_{i,k}(t,j)$	&   (function) O\&M variable cost	\\
%\end{tabular}

\smallskip
% Most of the remaining quantities of interest can be found using the balance of
% assets and a set of coefficients, e.g.\ coefficients for the fuel, emissions,
% costs, etc. In the framework, the coefficients are handled by functions of time
% and initial age. 
%
\paragraph{Effective capacity.}
Because decisions are made at the beginning of each period, only the remainder
of capacity can be used for production, this is there is an `effective'
capacity. These are encapsulated into variables and are
useful to define other relevant terms from the model. 
Let $W^t_{i,j}$, $Z^t_{i,k,j}$, and $X^t_{i,k,j}$ be the effective capacity for
existing, retrofitted, and new assets respectively, these are equated to the
right-hand-sides of Equations~\eqref{eq:wbal} through~\eqref{eq:xbal}, thus
defining the following constraints, 
%h why are these required
\begin{equation}
  W^t_{i,j} = w^{t}_{i,j} - u^{t}_{i,j} - \sum_{k \in K_i} y^{t}_{i, k,j}, 
  \; \forall t \in T, i \in I, j \in N_i,
\label{eq:weff}
\end{equation}
%
\begin{equation}
    Z^t_{i,k,j} = z^{t}_{i,k,j} - \overline{u}^t_{i,k,j} + 
    y^{t-\overline{\Delta}_{i,k}}_{i,k,j},\;
    \forall 
    t \in T, i \in I, k \in K_i, j \in N_i,
\end{equation}
%
and,
%
\begin{equation}
    X^t_{i,k,j} = x^{t}_{i,k,j} - v^t_{i,k,j} 
    ,\; \forall 
    t,j \in T, i \in I, k \in \tilde{K}_i.
\end{equation}
%
\paragraph{Production.}
%
This set of auxiliary variables
combines the effects of capacity and capacity factors resulting from the asset
mix. These represent the amounts of commodities generated in the context, e.g.\
MWh of electric power, tons of ammonia, etc. These are given by the following
constraints,
%
%The effective capacity can then be used to compute the critical quantities of
%the system. Notably the production quantities, e.g.\ in electrical sector, the
%power generation.  These quantities are contingent upon capacity factors and the
%length of the period, and they are presented in equation form, i.e., 
% 
%\begin{equation}
%    \mathtt{wGen}^t_{i,j} = \mathtt{kYrHr} \cdot \mathtt{cFact}_{i,0}\, 
%    W^t_{i,j},
%    \; \forall t \in T, i \in I, j \in N_i,
%\end{equation}
%
\begin{equation}
    \mathtt{wProd}^t_{i,j} = 
    \mathtt{kTime} \cdot 
    \mathtt{cFact}_{i,0} \cdot
    \mathtt{fCap}_{i,0}\left(t,j\right) \,
    W^t_{i,j},
    \; \forall t \in T, i \in I, j \in N_i,
\end{equation}
\begin{equation}
    \mathtt{zProd}^t_{i,j} = 
    \mathtt{kTime} \cdot 
    \mathtt{cFact}_{i,k} \cdot
    \mathtt{fCap}_{i,k}\left(t,j\right) \,
    Z^t_{i,j},
    \; \forall t \in T, i \in I, k\in K_i, j \in N_i,
    \label{eq:zprod}
\end{equation}
and
%
\begin{equation}
    \mathtt{xProd}^t_{i,k,j} = 
    \mathtt{kTime} \cdot 
    \mathtt{cFact}_{i,j}\, 
    \mathtt{fCap}_{i,k}\left(t,j\right) \,
    X^t_{i,j}
    ,\; \forall 
    t,j \in T, i \in I, k \in \tilde{K}_i,
\end{equation}
%
where $\mathtt{cFact}_{i,j}$ is the average capacity factor for technology $i$
and year $j$, $\mathtt{fCap}_{i,k}\left(t,k\right)$ is a function that returns
the capacity per asset unit (if the assets are integer, otherwise 1), 
and $\mathtt{kTime}$ is a time unit scale factor. 

%
\paragraph{Heat requirement.}
Heat is a critical quantity from which the fuel and emissions can be estimated.
These are related to the production quantities and coefficients that depend on
the age of the asset. 
The heat terms exists for existing, retrofitted, and new assets, however for
brevity, only the equation for existing asset heat is presented as follows, 
%
\begin{equation}
    \mathtt{wHeat}^t_{i,j} = \mathtt{fHeatRate}_{i,0}\! \left(t,j\right)\,
    \mathtt{wProd}^t_{i,j},
    \; \forall t \in T, i \in I_F, j \in N_i,
    \label{eq:fhr}
\end{equation}
%
where $\mathtt{fHeatRate}_{i,0}\! \left(t,j\right)$ is a function that returns
the heat required per unit produced. The same methodology extends to
$\mathtt{zHeat}^t_{i,j,k}$, and $\mathtt{xHeat}^t_{i,j,k}$. 

%
\paragraph{Carbon dioxide.}
The heat requirement is thus used to compute the carbon dioxide generation
alongside the carbon intensity factor $\mathtt{kCarbInt}_{i,k}$, as shown next,
\footnote{Equations for retrofitted and new assets are not written here for
brevity, nevertheless terms $\mathtt{zCo}^t_{i,j,k}$, $\mathtt{zCo}^t_{i,j,k}$, 
$\mathtt{zFueC}^t_{i,k}$, $\mathtt{zFuel}^t_{i,k}$, and the respective O\&M
variables do exists and the equation follow patterns similar as the ones for the
\texttt{w}-variables.} 
%
\begin{equation}
    \mathtt{wCo}^t_{i,j} = \mathtt{kCarbInt}_{i,0}\
    \mathtt{wHeat}^t_{i,j},
    \; \forall t \in T, i \in I_C, j \in N_i.
\end{equation}
%
\paragraph{Fuel cost.}
Fuel-based technologies have an associated cost of the fuel used for heating,
this is given here in terms of the heat required and a function
$\mathtt{fFuelC}_{i,0}\! \left(t,j\right)$, which returns the cost per heat unit
as function of time and initial age. Then the fuel cost equation is written as
follows,
%
\begin{equation}
    \mathtt{wFuelC}^t_{i} = \sum_{j\in N_i}
    \mathtt{fFuelC}_{i,0}\! \left(t,j\right)\,
    \mathtt{wHeat}^t_{i,j},
    \; \forall t \in T, i \in I_F. 
    %j \in N_i,
\end{equation}
%
\paragraph{O\&M cost.}
%fixed O\&M cost for technology $i$
%
Typically, O\&M cost has two components, i.e.\ fixed and variable. Similarly to
previous terms, these have a coefficient component which is returned from an
auxiliary function. Fixed cost is given as follows, 
\begin{equation}
    \mathtt{wFixOnM}^t_i = \sum_{j\in N_i} 
    \mathtt{fFixCost}_{i,0}\! \left(t,j\right)\,
    W^t_{i,j}.
    \; \forall t \in T, i \in I, 
\end{equation}
It is noted that the costs functions have a built-in adjustment component for
the cash flows as function of time (see Appendix).
%
%and variable O\&M cost for technology $i$,
The variable cost has a capacity factor component implicit in the production
variable, e.g.\ $\mathtt{wProd}^t_{i,j}$, and it is given as follows,
%
\begin{equation}
    \mathtt{wVarOnM}^t_{i} = \sum_{j\in N_i}
    \mathtt{fVarCost}_{i,0}\! \left(t,j\right)\,
    \mathtt{wProd}^t_{i,j},
    \; \forall t \in T, i \in I.
\end{equation}
%
\paragraph{Capital cost.}
%
The last of the cost components is the overnight capital cost associated with
retrofitted and new assets. 
For retrofits, these cost are defined in terms of the transition
from existing to retrofitted, $y^t_{i,k,j}$ as follows,
%
\begin{equation}
    \mathtt{zCap}^t_{i,k} = 
    \mathtt{fCapCost}_{i,k}\! \left(t\right)\,
    \sum_{j\in N_i}
    y^t_{i,k,j}
    \; \forall t \in T, i \in I, k \in  K_i.
\end{equation}
%
On the other hand, the capital cost of new allocations depends on the allocation
variable $\tilde{x}^t_{i,k}$, 
%
\begin{equation}
    \mathtt{xCap}^t_{i,k} = 
    \mathtt{fCapCost}_{i,k}\! \left(t\right)\,
    \tilde{x}^t_{i,k}
    \; \forall t \in T, i \in I, k \in \tilde{K}_i.
\end{equation}
%Most of these follow a pattern of explicit variable to right-hand-side 
%equation, i.e.
%`\texttt{variable0} \emph{equals} \texttt{constant} 
%\emph{times} \texttt{variable1}', where the \texttt{constant} term may be a
%function of $\left(t,j\right)$. Any additional constraint of the case study can
%be represented in terms of these quantities. For instance, the aggregate
%production
%quantities should cover the demand, i.e.,
%
\paragraph{Demand.}
The allocation of assets is driven by the production of commodities, e.g.\
electric power, such that demand over time is met. For this the production by
all assets in the model are aggregated such that they overtake the demand,
i.e.\
\begin{equation}
    \sum_{i\in I} \sum_{j\in N_i} \mathtt{wProd}^t_{i,j} 
    + \sum_{i \in I} \sum_{k \in K_i} \sum_{j\in N_j} \mathtt{zProd}^t_{i,k} 
    + \sum_{i \in I} \sum_{k \in \tilde{K}_i} \sum_{j\in T} 
    \mathtt{xProd}^t_{i,k} 
    \geq \mathtt{demand}^t,
    \; \forall t \in T,,
\end{equation}
%
where $\mathtt{demand}^t$ represents the demand at time $t$. 
%
\paragraph{Carbon budget constraint.}
Another driving aspect of the system is the limits on the emissions of carbon
dioxide of the system. The form of this constraint depends on the context; here
a upper limit on the aggregate carbon emission for the horizon is set, i.e.\
\begin{equation}
    \sum_{t\in T} \sum_{i\in I_C} \sum_{j\in N_i} \mathtt{wCo}^t_{i,j} 
    + \sum_{t\in T} \sum_{i \in I_C} \sum_{k \in K_i} \sum_{j\in N_j}
    \mathtt{zCo}^t_{i,k} 
    + \sum_{t\in T} \sum_{i \in I_C} \sum_{k \in \tilde{K}_i} \sum_{j\in T} 
    \mathtt{xCo}^t_{i,k} 
    \leq \mathtt{co2Budget}.
   % \; \forall t \in T.
    \label{eq:co2budgeteq}
\end{equation}

\paragraph{Net present value}
The net present value one of the elements of the  objective function of the
problem. It contains discounted costs over time as given by the coefficient
functions previously describe. The main components are overnight capital, O\&M,
fuel and retirement cost for all assets. It is given as follows,
\begin{equation}
    \begin{split}
        \mathtt{NPV} &= \\
        % 0
        & \sum_{t\in T} \sum_{i\in I} 
        \mathtt{wVarOnM}^t_{i}
        +\mathtt{wFixOnM}^t_{i} \\
        % 1
        &+ \sum_{t\in T} \sum_{i\in I} \sum_{k\in K_i} 
        \mathtt{zCap}^t_{i,k}
        + \mathtt{zFixOnM}^t_{i,k} 
        + \mathtt{zVarOnM}^t_{i,k} \\
        % 2
        &+ \sum_{t\in T} \sum_{i\in I} \sum_{k\in \tilde{K}_i} 
        \mathtt{xCap}^t_{i,k} 
        + \mathtt{xFixOnM}^t_{i,k} 
        + \mathtt{xVarOnM}^t_{i,k} \\
        % 3
        &+ \sum_{t\in T} \sum_{i\in I_F} 
        \mathtt{wFuelC}^t_{i} 
        + \sum_{k\in K_i} \mathtt{zFuelC}^t_{i,k} 
        + \sum_{k\in \tilde{K}_i} \mathtt{xFuelC}^t_{i,k}   \\
        % 4
        &+ \sum_{t\in T} \sum_{i\in I} 
        \sum_{j\in N_i} \mathtt{wRetC}^t_{i,j} 
        + \sum_{k\in K_i} \sum_{j\in N_i} \mathtt{zRetC}^t_{i,k} 
        + \sum_{k\in \tilde{K}_i} \sum_{j\in T} \mathtt{xRetC}^t_{i,k}.   \\
    \end{split}
\end{equation}

\paragraph{Overall terminal cost}
The second element of the objective is the terminal cost. This encapsulates the
contributions of all the assets, 
\begin{equation}
    \mathtt{termCost} = 
    \sum_{i\in I} \sum_{j\in N_i} \mathtt{wTermC}_{i,j}
    + \sum_{i\in I} \sum_{k\in K_i} \sum_{j\in N_i} \mathtt{zTermC}_{i,k,j}
    + \sum_{i\in I} \sum_{k\in \tilde{K}_i} \sum_{j\in T} 
    \mathtt{zTermC}_{i,k,j}.
\end{equation}

\paragraph{Soft service life penalty}
Lastly, the soft retirement cost penalty, which for the objective presented
subsequently, is given in costs units,
\begin{equation}
    \mathtt{softSl} = \mathtt{wSoftSl} + \mathtt{zSoftSl} + \mathtt{xSoftSl}.
\end{equation}

\subsubsection{Objective function}
The objective function of the problem is the minimization of the net present
value, with the terminal and soft service life terms. 
Let $\mathbf{x}$ represent a column vector that concatenates all the variables
of the problem, and $\mathcal{X}$ represent the feasible set as given by all the
constraints described thus far. Then the objective function $O(\mathbf{x})$ is
to be minimized as follows, 
%
\begin{equation}
    \begin{split}
        \min\; O\left(\mathbf{x}\right):= \alpha \, \mathtt{NPV} 
        + \beta \, \mathtt{termCost} 
        + \gamma \, \mathtt{softSl}, \; \text{s.t.} \; \mathbf{x} \in
        \mathcal{X}.
    \end{split}
\end{equation}
%
The terms in the objective have scalars $\alpha, \beta, \gamma \in \mathbb{R}_+$
whose value are arbitrarily set. 
%
\subsection{Implementation}
%%%%%%%%
% the model has been written in julia
% a flexible representation of a sector?
% data requirements?
% objective function value
%%%%%%%%%
The collection of terms and ideas was translated into a codebase that provides
the grounds for deployment in the \texttt{Julia} 
\cite{bezanson2012julia} programming
language. This language was selected because of performance, multiplatform
availability, and the extensive supporting libraries, most notably the modelling
language for mathematical programming, \texttt{JuMP} \cite{dunning2017jump}.
\texttt{JuMP} provides the basis for the translation of the model into code, and
it also interfaces with several open-source optimization solvers. Therefore,
\texttt{Julia} provides the full scope of possible operations from data
input-output to processing and solution. 
%
% sectors are not one-dimensional, they have multi-layered, thus we require a
% methodology to capture this
Data is provided to the framework through spreadsheets, which are translated to
arrays, e.g.\ the initial capacity $\mathtt{initCap}_{i,j}$ for technologies
$i\in I$ and initial age $j\in N_i$. Modelling requirements not covered in this
report can be added at either pre- or post-generation model stages, provided
they do not contradict the core assumptions.
%
%
%
\begin{appendices}
\section{Auxiliary functions and notes}\label{appendix}
    %\todo[inline]{Write short notes on these functions.}
    \paragraph{Discount rate function.}
%------------------------------------------------------------------------------
This function utilizes a discount rate parameter $\mathtt{discRate}_i$, which
needs not be the same for all species.
\begin{equation}
    \mathtt{fDscnt}_i
    \left(t\right)
    := {\left(1+\mathtt{discRate}_i\right)}^{-t}.
\end{equation}
%
    \paragraph{Retirement cost function}

%
%------------------------------------------------------------------------------
The retirement cost is an important quantity associated with the retirement
decision variables. This has two components, the outstanding loan amount, and
the lost of revenue. The cost is also adjusted by time and is given by the 
following function of time and proxy of age,
%
\begin{equation}
    \mathtt{fRetCost}_{i,k}\!\left(t,j\right):= 
    \mathtt{fDscnt}_i\!\left(t\right) \cdot
    \left(
    \mathtt{fOutLoan}_{i,k}\!\left(t,j\right) +
    \mathtt{fLostSale}_i\!\left(t,j\right)
    \right),
\end{equation}
%
Both components are also given as functions of time and proxy of age, and they
are defined as piecewise continuous functions. For the outstanding loan, the
loan period parameter, $\mathtt{loanP}_i$, is used to create the following
function,
%
\begin{equation}
    \mathtt{fOutLoan}_{i,k}\!\left(t,j\right) :=
    \left\{
        \begin{matrix}
            \frac{\left(\mathtt{loanP}_i - \mathtt{age}_{t,j}\right)}
            {\mathtt{loanP}_i} 
            %\mathtt{fDisc}\left(t\right) 
            \mathtt{kCc}_{i,k} \mathtt{fCapCost}_{i,j}(t), & 
            \text{if}\; \mathtt{loanP}_i-\mathtt{age}_{t,j}\geq 0 ,\\
            0, & \text{otherwise}.
        \end{matrix}
    \right.
\end{equation}
%
Similarly the lost of revenue requires the plant service life, 
$\mathtt{sLife}_i$, and the price of electricity, $\mathtt{elecSale}$. These
are used in the following function,
%
\begin{equation}
    \mathtt{fLostSale}_i\!\left(t,j\right) :=
    \left\{
        \begin{matrix}
            \left(\mathtt{sLife}_i - \mathtt{age}_{t,j}\right) 
            \mathtt{elecSale}_t, & 
            \text{if}\; \mathtt{sLife}_i-\mathtt{age}_{t,j} \geq 0 ,\\
            0, & \text{otherwise}.
        \end{matrix}
    \right.
\end{equation}
%------------------------------------------------------------------------------
%
    \paragraph{Heat rate and fuel cost}
%
The heat rate is assumed to increase with time given by $\mathtt{hRinc}$, and 
both functions are given in terms of time and proxy of age as follows,
\begin{equation}
    \mathtt{fHeatRate}_{i,k}\!\left(t,j\right) := 
    \mathtt{kHr}_{i,k} \cdot
    \mathtt{heatRate}_{i,j} \cdot {\left(1+\mathtt{hRinc}\right)}^t
    \label{eq:f_hr},
\end{equation}
%
\begin{equation}
    \mathtt{fFuelC}_{i,0}\! \left(t,j\right) :=
    \mathtt{fDscnt}_i\! \left(t\right) \cdot 
    \mathtt{kFu}_{i,k} \cdot
    \mathtt{fuelC}_{i,t}. 
\end{equation}
%
    \paragraph{Soft service life penalty parameter function.}
%
%
The soft service life penalty parameter function is primarily required to
increase monotonically. Moreover, the rate at which the returned value increases
can be set accordingly to the attitude towards the late assets. The following
function was identified to capture the requirements previously mentioned,
%
\begin{equation}
    \mathtt{fPenalty}_i \left(\mathtt{age}\right) = 
    \mathtt{kPen}_i \cdot \exp
    \left(
    \left(\mathtt{age}-\mathtt{sLife}_i\right)/\mathtt{sLife}_i
    \right),
\end{equation}
%
where $\mathtt{sLife}_i$ is the service life for technology kind $i$, and
$\mathtt{kPen}_i$ is an arbitrary scaling value. This function will have 
an aggressive response as the asset ages, as opposed to, for instance, a linear
function of the relative age. $\mathtt{kPen}_i \in \mathbb{R}_{\geq 0}$ must be
selected keeping into account the overall scaling of the problem, and the
aggressiveness of the retirement penalty.
%
    \paragraph{Carbon capture retrofits heat rate and power output reduction.}
%------------------------------------------------------------------------------
The carbon capture retrofits of existing coal and natural gas combined cycle 
power plants reduce the energy efficiency of base plants in a nontrivial way.
For the purpose of this manuscript, the side effects of this kind of retrofit
follow the findings of \cite{supekar2017sourcing} in a simplified way.
Two effects are considered, a) the reduction of the heat rate, and b)
the reduction of the power output. Both effects require a \%-points of 
efficiency decrease, which can be computed using the methodology found in
\cite{supekar2017sourcing}. Nevertheless, for a) the \%-points of efficiency 
are directly subtracted from the the base heat rate ($\mathtt{heatRate}$) in 
Equation~\eqref{eq:f_hr}; and for b) the \%-points are used to create
a reduction multiplier, which is embedded in Equation~\eqref{eq:zprod}.
%
\end{appendices}
%
% we need a table for:
% sets, variables, functions/constants
% symbol \t description
% 
%
\end{document}
%
