\documentclass{amsart}

\usepackage[english]{babel}
\usepackage{amsmath}
\usepackage{bm}


\usepackage[letterpaper, margin=1in]{geometry}

\usepackage{float}
\usepackage{tabularx}
\usepackage{longtable}
\usepackage{hyperref}

%
\DeclareFontFamily{U}{matha}{\hyphenchar\font45}
\DeclareFontShape{U}{matha}{m}{n}{ <-6> matha5 <6-7> matha6 <7-8>
matha7 <8-9> matha8 <9-10> matha9 <10-12> matha10 <12-> matha12 }{}
\DeclareSymbolFont{matha}{U}{matha}{m}{n}
\DeclareMathSymbol{\myvee}{\mathbin}{matha}{"5F}
\newcommand{\myveebar}{\mathbin{\vcenter{\hbox{$ \underline{\mkern-2.4mu{\myvee}\mkern-2.4mu} $}}}}

%
\begin{document}

\title{\texttt{DRE4M} discrete version equations.}
\author{David Thierry \{dthierry@anl.gov\}}

\date{\today}
\maketitle

\section{Introduction.}
%
The discrete version of the model maintains variables that are a combination of
continuous and discrete. Continuous variables are associated with quantities
that can be measured, e.g.\ plant capacity, whereas discrete variables are
associated with a state or transition of the plant. 
%
In the model both kinds of variables are mainly indexed by time and
location, and subsequently by technology, e.g.\ retrofit. 
%
Time is sliced into periods of equal size, and similarly each one of them
divided into sub-periods. Sets are described in Table~\ref{tab:sets}.
%
\begin{table}[h]
    \caption{Sets.}
    \begin{tabular}{@{}ll@{}}
        $\mathcal{K}$ & \textit{Generic} technology set,\\
        $\mathcal{F}$ & Fuels set.\\
        $\mathcal{F}_1$	& Feed-stocks set.\\
        $\mathcal{L}$ & Locations set,\\
        $\mathcal{N}$ & New plant set $\left\{1,2,\dots,\mathtt{Nnew}\right\}$,\\
        $\mathcal{P}$	& Period set $\left\{1,2,\dots,\mathtt{Nper}\right\}$, \\
        $\mathcal{P}_1$	& Sub-period set $\left\{1,2,\dots,\mathtt{NsubPer}\right\}$,\\
        $\mathcal{R}$ & Retrofit set $\left\{1,2,\dots,\mathtt{Nretrf}\right\}$,\\
    \end{tabular}\label{tab:sets}
\end{table}
%

Variables are usually written in \texttt{monospace} style, and have a prefix.
Four types of prefixes are used throughout this manuscript \texttt{O},
\texttt{R}, \texttt{N}, and \texttt{E}, which signal online, retrofit, new, and
expansion state. These are listed in Table~\ref{tab:vars}. Parameters follow
similar notation, except when they apply globally and thus no prefix is
specified, e.g.\ \texttt{CelecR} the cost of electricity, which applies to all
plants regardless of kind. Parameters are thus listed in Table~\ref{tab:params}. 
%
\begin{longtable}[c]{@{}ll@{}}
 \caption{Variables.\label{tab:vars}}\\
    $\mathtt{annCost}$	& Annuity, \\
    $\mathtt{cap}$	& Plant capacity, \\
    $\mathtt{cOnM}$	& O\&M cost, \\
    $c_e$	& Expansion capacity, \\
    $\mathtt{EannCost}$ & Exp.\ annuity cost, \\
    $\mathtt{EcOnM}$ & Exp. O\&M cost, \\
    $\mathtt{eh}$	& Heat, \\
    $\mathtt{ehOns}$	& Heat on-site (elec.), \\
    $\mathtt{ehCo2}$	& Fuel CO2, \\
    $\mathtt{ehOnsCo2}$	& Fuel CO2 (elec.), \\
    $\mathtt{EovNc}$ & Exp.\ overnight cap.\ cost, \\
    $\mathtt{ep0}$	& Total plant CO2, \\
    $\mathtt{ep1gce}$	& Captured-emitted CO2, \\
    $\mathtt{ep1gcs}$	& Captured-stored CO2, \\
    $\mathtt{fstk}$	& Feedstock, \\
    $\mathtt{fu}$	& Fuel, \\
    $\mathtt{fuOns}$	& Fuel on-site (elec.), \\
    $\mathtt{NannCost}$ & New annuity cost, \\
    $\mathtt{Nc0}$	& New-plant capacity, \\
    $\mathtt{Ncap}$ & New cap., \\
    $\mathtt{NcapD}$ & New disaggregated cap., \\
    $\mathtt{NcOnM}$ & New plant O\&M payment, \\
    $\mathtt{NcFuel}$ & New plant fuel cost, \\
    $\mathtt{NcElec}$ & New plant fuel cost, \\
    $\mathtt{NcCcus}$ & New plant trans.\& storage cost, \\
    $\mathtt{Nep1ge}$ & New CO2 emitted, \\
    $\mathtt{Nloan}$	& New plant loan balance, \\
    $\mathtt{Nladd}$	& New plant added loan, \\
    $\mathtt{NovNc}$ & New overnight cost, \\
    $\mathtt{Npay}$	& New plant loan payment, \\
    $\mathtt{OcElec}$ & Existing elec.\ cost, \\
    $\mathtt{OcFuel}$ & Existing plant fuel cost, \\
    $\mathtt{Ocap}$ & Existing cap., \\
    $\mathtt{OcapRd}$ & Existing disaggregated cap., \\
    $\mathtt{OcCcus}$ & Existing CO2 trans.\& storage cost, \\
    $\mathtt{OcOnM}$ & Existing plant O\&M payment, \\
    $\mathtt{Oep1ge}$ & Existing CO2 emitted, \\
    $\mathtt{Oep1gcs}$ & Existing CO2 stored, \\
    $\mathtt{Ofu}$ & Existing fuel, \\
    $\mathtt{Opay}$	& Existing plant loan payment, \\
    $\mathtt{OretCost}$	& Existing ret.\ cost, \\
    $\mathtt{Ou}$ & Existing elec., \\
    $\mathtt{OvNc}$	& Overnight capital cost, \\
    $\mathtt{prCo2}$	& Process CO2, \\
    $\mathtt{RloanP}$	& Retrofit loan balance (positive), \\
    $\mathtt{RloanP}$	& Retrofit loan balance (negative), \\
    $\mathtt{u}$	& Electricity, \\
    $\mathtt{uOns}$	& Elec.\ on-site, \\
    $x$	& Generic vector of variables, \\
    $x^e$	& Expansion vector of variables, \\
    $x_n$	& New-plant vector of variables, \\
    $x_o$	& Online vector of variables, \\
    $x_r$	& Retrofit vector of variables, \\
    $\nu$	& Generic disaggregated vector of variables, \\
 \end{longtable}
%
\begin{longtable}[c]{@{}ll@{}}
    \caption{Parameters.}\label{tab:params} \\
    $A$ & Generic coefficient matrix,\\
    $A^e$	& Expansion coef.\ matrix, \\
    $A^n$	& New-plant coef.\ matrix, \\
    $A^r$	& Retrofit coef.\ matrix, \\
    $\mathtt{annF}$	& Annuity factor, \\
    $b$ & Generic rhs.\ vector,\\
    $b^e$	& Expansion rhs.\ vector, \\
    $b^r$	& Retrofit rhs.\ vector, \\
    $b^n$	& New-plant rhs.\ vector, \\
    $\mathtt{cOnMr}$	& O\&M cost rate, \\
    $\mathtt{cOnMrhs}$	& O\&M cost rate rhs., \\
    $\mathtt{c0}$	& Initial capacity, \\
    $\mathtt{CccusR}$ & CO2 transport and storage cost rate, \\
    $\mathtt{CfuelR}$ & Fuel cost rate, \\
    $\mathtt{CelecR}$ & Electricity cost rate, \\
    $\mathtt{CO2Bound}$ & CO2 Bound, \\
    $\mathtt{demand}$ & Demand, \\
    $\mathtt{EannF}$	& Exp.\ annuity fact., \\
    $\mathtt{ehOnsRhs}$	& On-site elec.-to-heat rate rhs., \\
    $\mathtt{EovNcF}$	& Exp.\ overn.\ cap.\ cost factor, \\
    $\mathtt{EovNcRhs}$	& Exp.\ overn.\ cap.\ cost rhs., \\
    $\mathtt{Femf}$	& Fuel emission factor, \\
    $\mathtt{Ffr}$	& Fuel fraction, \\
    $\mathtt{FfrOns}$	& On-site fuel fraction, \\
    $\mathtt{fstkR}$	& Feedstock rate, \\
    $\mathtt{fstkRhs}$	& Feedstock rate rhs., \\
    $\mathtt{Hi}$	& Heat intensity, \\
    $\mathtt{HiRhs}$	& Heat intensity rhs., \\
    $H$ & Logic relations coeff.\ matrix,\\
    $h$ & Logic relations rhs.\ vector,\\
    $\mathtt{Nper}$	& Number of periods, \\
    $\mathtt{NsubPer}$	& Number of sub-periods, \\
    $\mathtt{Nretrf}$	& Number of retrofit, \\
    $\mathtt{Nc0}^{UB}$	& New-plant capacity upper bound, \\
    $\mathtt{Nc0}^{LB}$	& New-plant capacity lower bound, \\
    $\mathtt{msUB}$ & Market share fraction, \\
    $\mathtt{PrCoR}$	& Process CO2 rate, \\
    $\mathtt{PrCoRhs}$	& Process CO2 rhs, \\
    $\mathtt{OnsHr}$	& On-site elec.-to-heat rate, \\
    $\mathtt{ovNcF}$	& Overnight capital cost factor, \\
    $\mathtt{ovNcRhs}$	& Overnight capital cost rhs, \\
    $\mathtt{Ui}$	& Electricity intensity, \\
    $\mathtt{UiRhs}$	& Electricity int.\ rhs., \\
    $\mathtt{uOnsF}$	& On-site elec.\ factor, \\
    $x^{\text{LB}}$ & Generic vector lower bound,\\
    $x^{\text{UB}}$ & Generic vector upper bound,\\
    $r$	& interest rate, \\
    $\chi$	& Capture fraction, \\
    $\sigma$	& Storage fraction, \\
    $\delta$ & Discount factor, \\
\end{longtable}
%
Finally, as described in subsequent sections, discrete decisions are expressed
with \emph{True, False} (Boolean) valued variables, which can be translated into
\emph{0, 1} (Binary) variables. These are listed in Table~\ref{tab:bools}.
%
\begin{table}[h]
    \caption{Discrete variables Boolean // binary.}
    \begin{tabular}{@{}lll@{}}
        $Y$ & $y$ & Generic variable, \\
        $\bm{Y}$ & $\bm{y}$ & Vector of concatenated disc.\ vars., \\
        $Y^e$ & $y^e$ & Expansion state, \\
        $\check{Y}^e$ & $\check{y}^e$ & Expansion activation, \\
        $Y^n$ & $y^n$ & New-plant state, \\
        $\check{Y}^n$ & $\check{y}^n$ & New-plant activation, \\
        $Y^o$ & $y^o$ & Generic Boolean/Discrete, \\
        $\hat{Y}^o$ & $\hat{y}^o$ & Online deactivation, \\
        $Y^r$ & $y^r$ & Retrofit state, \\
        $\check{Y}^r$ & $\check{y}^r$ & Retrofit activation, \\
        $\mathtt{nYps}$ & $\mathtt{nyps}$ & New payment state, \\
        $\mathtt{rYps}$ & $\mathtt{ryps}$ & Retrofit payment state, \\
    \end{tabular}\label{tab:bools}
\end{table}
%

\section{Characteristic system (no expansion).}\label{sec:char_sys}
%
It is assumed that most quantities relevant to the plant, e.g.\ heat,
electricity, etc.\ are either \emph{linear} function of a few key variables,
e.g.\ capacity, or are derived from other, e.g.\ fuel emission which is derived
from fuel consumed.  Let us consider the plant capacity $\text{cap}_{i,j,l}$ for
an arbitrary period, sub-period, and location.  For this section it is assumed
that no expansion is executed on the plant, i.e.
$\text{cap}_{i,j,l}=\text{c0}_{l}$.  Then, subset of the variables is directly
associated with capacity as follows:
%
\begin{align}\label{eq:eqset1}
    \mathtt{eh}_{ijl} &= \mathtt{Hi}_{ijl} \mathtt{cap}_{ijl} 
    + \mathtt{hiRhs}_{ijl}, \\
    %
    \mathtt{u}_{ijl}  &= 
    \mathtt{Ui}_{ijl}\left(1-\mathtt{uOnsF}\right)\mathtt{cap}_{ijl} 
    + \mathtt{UiRhs}_{ijl} \left(1 - \mathtt{uOnsF}\right), \\
    % onsite elec
    \mathtt{uOns}_{ijl}  &= 
    \mathtt{Ui}_{ijl}  \mathtt{uOnsF} \; 
    \mathtt{cap}_{i,j,l} + \mathtt{uRhs}_{i,j,l} \mathtt{uOnsF}, \\
    %
    \mathtt{fstk}_{ijlf}  &= 
    \mathtt{fstkR}_{ijlf}  
    \mathtt{cap}_{ijl} + \mathtt{fstkRhs}_{ijlf},\; \forall f \in \mathcal{F}_1, 
    \\
    %
    \mathtt{prCo2}_{i,j,l} &= \mathtt{PrCoR}_{i,j,l} \mathtt{cap}_{i,j,l} 
    + \mathtt{PrCoRhs}_{i,j,l}, \\
    %
    \mathtt{cOnM}_{i,j,l} &= 
    \mathtt{cOnMr}_{i,j,l} \mathtt{cap}_{i,j,l} +
    \mathtt{cOnMrhs}_{i,j,l},
    \\
    %
    \mathtt{ovNc}_{i,j,l} &= \mathtt{OvNcF}_{i,j,l} \mathtt{cap}_{i,j,l}
    + \mathtt{OvNcRhs}_{i,j,l}, \\
    & \forall \; i \in \mathcal{P},j \in \mathcal{P}_1 ,
    l \in \mathcal{L}, \notag
\end{align}
%
where the variables on the left-hand-sides are respectively, heat, electricity,
on-site electricity, feed-stocks, non-fuel emission, O\&M cost, and overnight
capital cost.  Furthermore, the coefficients on the right are the per unit of
capacity rates, and the ``right-hand-side''\footnote{In most cases these
right-hand-sides are equal to zero.}.
%
The previous quantities then enable the calculation of the remaining variables
in the following way,
%
\begin{align}\label{eq:eqset2}
    \mathtt{fu}_{i,j,l,f} &= 
    \mathtt{Ffr}_{i,j,l,f} \mathtt{eh}_{i,j,l}, \\
    %
    \mathtt{ehOns}_{i,j,l} &= 
    \mathtt{OnsHr}_{i,j,l} \mathtt{uOns}_{i,j,l} + \mathtt{ehOnsRhs}_{i,j,l}, \\
    %
    \mathtt{fuOns}_{i,j,l,f} &= 
    \mathtt{FfrOns}_{i,j,l,f} \mathtt{ehOns}_{i,j,l}, \; \forall f \in
    \mathcal{F}, \\
    % fuel em
    \mathtt{ehCo2}_{i,j,l} &= 
    \sum_{f \in \mathcal{F}} \mathtt{Femf}_{f} \mathtt{eh}_{i,j,l,f}, \\
    % fuel-onsite-elec em
    \mathtt{ehOnsCo2}_{i,j,l} &= 
    \sum_{f \in \mathcal{F}} \mathtt{Femf}_{f} \mathtt{ehOns}_{i,j,l,f}  , \\
    % ep0
    \mathtt{ep0}_{i,j,l} &= \mathtt{prCo2}_{i,j,l} 
    + \mathtt{ehCo2}_{i,j,l}
    + \mathtt{ehOnsCo2}_{i,j,l}, \\
    % ep1ge
    \mathtt{ep1ge}_{i,j,l} &= \left(1-\chi_{l}\right) \mathtt{ep0}_{i,j,l}, \\
    %
    \mathtt{ep1gce}_{i,j,l} &= 
    \chi_{l} \left(1-\sigma_{l}\right) \mathtt{ep0}_{i,j,l}, \\
    %
    \mathtt{ep1gcs}_{i,j,l} &= 
    \chi_{l} \sigma_{l} \mathtt{ep0}_{i,j,l}, \\
    %
    \mathtt{annCost}_{i,j,l} &= 
    \mathtt{annF}_{i,j,l}\; \mathtt{nvNc}_{i,j,l}, \\
    & \forall \; i \in \mathcal{P},j \in \mathcal{P}_1 ,
    l \in \mathcal{L}. \notag
\end{align}
%
These equations represent: fuel, on-site fuel (elec.\ gen.), fuel CO2 emission,
on-site fuel CO2 emission (elec.\ gen.), total CO2 emission, scope 1 emitted,
scope 1 captured/emitted, scope 1 captured stored, and annuity cost. 
%
The equations previously presented in this section can be condensed into a
linear system, where the coefficients are used to assemble a matrix $A_{i,j,l}$,
and a right-hand-side $b_{i,j,l}$. Let $x_{i,j,l}$  a variable vector, where
each element comes from the previous equations. Then, the equations are
condensed into the system,
%
\begin{equation}\label{eq:base_sys}
    A_{i,j,l} x_{i,j,l} = b_{i,j,l},
    \; \forall \; i \in \mathcal{P},j \in \mathcal{P}_1 , l \in \mathcal{L}.
\end{equation}
%
\section{Introduction to disjunctive constraints}\label{sec:DP}
%
Disjunctive Programming (DP) refers to linear programming with disjunctive
constraints. In the context of the present model, DP is used as in a twofold
way, (a) to represent the discrete nature of certain key decisions, for
instance, technology retrofit, and (b) to impose a policy for the possible
combinations of these decisions.  In this section a generic case is given, which
then referenced later on when discussing the specific discrete decisions of the
model. First, let $k\in K$ represent the disjunctive constraint index and $i\in
D_k$ be the disjunctive terms. Then the disjunctive constrains are represented
as,
%
\begin{equation}\label{eq:dpcon}
    \myveebar_{i \in D_k}
    \begin{bmatrix}
        Y_{ki} \\
        A_{ki} x \leq b_{ki}
    \end{bmatrix}, \; k\in K,
\end{equation}
%
where $Y_{ki}\in\{\text{True}, \text{False}\}$ is a Boolean variable, $A_{ki}\in
\mathbb{R}^{m\times n}$, and $b_{ki}\in \mathbb{R}^{m}$ are the coefficient
matrix and right-hand-side vector. Consider $k=1$, the constraint reads,
$Y_{1,1}$ \textbf{XOR} (exclusive or) $Y_{1,2}$ \textbf{XOR} $\dots Y_{1,i}$. 
Then, the associated linear system becomes active \emph{only} if its
corresponding Boolean variable is True, otherwise it not considered.
%
In order to use off-the-shelf optimization solvers the system must be
reformulated first. In general, Eq.\eqref{eq:dpcon} can always be written as a
Mixed-Integer Linear Program (MILP), where the discrete decision are encoded
with 1--0 variables as opposed to Boolean variables.  For instance, one
reformulation of system~\eqref{eq:dpcon} results in the following set of
constraints,
%
\begin{equation}\label{eq:convex_hull}
    \begin{split}
        \sum_{i\in D_k} \nu_{ki} &= x, \; k\in K, \\
        A_{ki} x_{ki} &= b_{ki} y_{ki}, \; k\in K, i\in D_k, \\
        \sum_{i\in D_k} y_{ki} &= 1, \; k\in K, \\
        x^{\text{LB}}y_{ki} &\leq \nu_{ki} \leq x^{\text{UB}}y_{ki},
        \; k\in K, i\in D_k\\
    \end{split}
\end{equation}
%
where $\nu_{ki} \in \mathbb{R}^n$, $y_{ki}\in \{0,1\}$, and $x^{\text{LB}}$ and 
$x^{\text{UB}}$ are the upper and lower bounds of the variable $x$.
Equation~\eqref{eq:convex_hull} is used in the context of the model in-code, but
Eq.~\eqref{eq:dpcon} is used for the rest of this document as shorthand and for
clarity.
%
The remaining element of this methodology are logic relations between logic
variables represented by the vector constraint, $\Omega \left(\bm{Y}\right) =
\text{True}$. These limit the set of possible combinations of Boolean variables
and encode specific rules that the discrete decisions must follow, e.g.\ ``a new
plant can only follow a retired plant''. For the MILP reformulation these are
expressed using a coefficient matrix $H$ and vector $h$ respectively in the
following equation,
%
\begin{equation}\label{eq:linear_logic}
    H \bm{y} \geq h.
\end{equation}
%
Finally, to achieve this constraint, one must transform the logical relations
by applying the De Morgan's Laws and distributive rules to obtain a
statement with conjunctions of clauses, i.e.\ the \emph{Conjunctive Normal Form}
(CNF) which is then transformed to a system of linear inequalities, i.e.\ the
matrices and vectors of Eq.~\eqref{eq:linear_logic}.
%
\section{Plant Expansion}
%
To satisfy the potential demand increase, plants can expand. In the model this
action has discrete and continuous elements. Firstly, let 
$Y^e_{ijl}\in\{\text{True}, \text{False}\}$ signal plant expansion if $Y^e_{ijl} 
= \text{True}$. Secondly, let $c_{e,i,j,l} \geq \mathbb{R}_{\geq 0}$ represent
the additional plant capacity. The reminder relevant quantities are associated
with the added capital and O\&M cost. 
If $Y^e_{ijl}$=True, the basis for plant expansion is given by the following
equation system,
%
\begin{align}\label{eq:exp_sys}
    \mathtt{EovNc}_{i,j,l} &= \mathtt{EovNcF}_{i,j,l} c_{e,i,j,l} 
    + \mathtt{EovNcRhs}_{i,j,l}, \\
    %
    \mathtt{EannCost}_{i,j,l} &= 
    \mathtt{EannF}_{i,j,l}\; \mathtt{EovNc}_{i,j,l},\\
    %
    \mathtt{EcOnM}_{i,j,l} &= 
    \mathtt{EcOnMr}_{i,j,l} c_{e,i,j,l} + \mathtt{EcOnMrRhs}_{i,j,l}, \\
    & \forall \; i \in \mathcal{P},j \in \mathcal{P}_1 ,
    l \in \mathcal{L}, \notag
    %
\end{align}
%
where the quantities on the left represent overnight capital, annual capital,
and O\&M costs of expansion. The coefficients are given by unit of expanded
capacity, e.g.\ $\mathtt{eOvNcF} \left[=\right] \text{USD}/(\text{tonne/year})$.
If no expansion is awarded, then $Y^e_{ijl}=\text{False}$ and all variables are
set to zero. In general this is all condensed into the following constraints,
%
\begin{equation}\label{eq:dp_exp}
    \begin{bmatrix}
        Y^e_{ijl} \\
        A^e_{ijl} x^e_{ijl} \leq b^e_{ijl}
    \end{bmatrix}
    \myveebar
    \begin{bmatrix}
        \neg Y^e_{ijl} \\
        x^e_{ijl} = 0
    \end{bmatrix},
    \; \forall \; i \in \mathcal{P},j \in \mathcal{P}_1 ,
    l \in \mathcal{L},
\end{equation}
%
where $x^e_{ijl}$ is vector that contain the concatenated variables from
Eq.~\eqref{eq:exp_sys}, and  $A^e_{ijl}$ and $b^e_{ijl}$ represent the
coefficient matrix and right-hand-side-matrices from the same equation. This
system is subsequently reformulated following Sec.~\ref{sec:DP}, after which
binary variables $y^e_{ijl}$ replace the Boolean variables.
Finally, the overall plant capacity is upgraded to have two components, the base
capacity $\text{c0}_l$ and the expanded capacity $c_{e,ijl}$, i.e.,
%
\begin{equation}
    \mathtt{cap}_{i,j,l} = \text{c0}_{l} + c_{e,i,j,l}.
    \; \forall \; i \in \mathcal{P},j \in \mathcal{P}_1 ,
    l \in \mathcal{L}.
\end{equation}
%
\section{Plant retrofit.}\label{sec:rf}
%
The plant retrofit model is constructed from Sec.~\ref{sec:char_sys}, and
Eq.~\eqref{eq:base_sys}, moreover it requires the inclusion of the set of
retrofits $\mathcal{R}$, and Boolean variables $Y^r_{ijlk} \in
\left\{\text{True, False}\right\}$ for $k\in \mathcal{R}$.
%
\begin{equation}\label{eq:rf}
    \myveebar_{k \in \mathcal{R}}
    \begin{bmatrix}
        Y^r_{ijlk} \\
        A^r_{ijlk} x^r_{ijl} \leq b^r_{ijlk}
    \end{bmatrix},
    \; \forall \; i \in \mathcal{P},j \in \mathcal{P}_1 ,
    l \in \mathcal{L}.
\end{equation}
%
This has the vector $x^r_{ijl}$ in which each element is a variable from the
equations Eqs.~\eqref{eq:eqset1}--\eqref{eq:eqset2}, e.g.\
$\left[\mathtt{Rcap}_{ijl}, \mathtt{Reh}_{ijl}, \mathtt{Ru}_{ijl}, \dots
\right]$ (prefix $\mathtt{R}$ denotes retrofit). Also, it requires matrices and
right-hand-sides for each retrofit $A^r_{ijlk}$ and $b^r_{ijlk}$. 
The constraint specifies that a single retrofit must be selected at a given time
and location.
Moreover, by assigning different values to the matrices and right-hand-side
vector this disjunction would allow to change the identity of the plant, for
instance changing the process emissions, or electricity consumption, etc. 
In addition to this constraint, it is noted that logical relations are necessary
to limit the number of retrofit transition, e.g.\ once for the whole horizon.
The logic relationships are described in a separate section. 
%
\section{New plant}
%
New plants use a near identical mathematical representation as the retrofits.
For this case, let the set of possible new plant be $\mathcal{N}$, and Boolean
variables $Y^n_{ijlk} \in \left\{\text{True, False}\right\}$ for $k\in
\mathcal{N}$. Unlike retrofits new plants have arbitrary capacity, which can
potentially allow for the creation of plants of sufficient capacity to clear
demand. For this it is considered a new capacity variable that with upper
and lower bounds,
%
\begin{equation}
    \mathtt{Nc0}^{\text{LB}}_l \leq \mathtt{Nc0}_{l} 
    \leq \mathtt{Nc0}^{\text{UB}}_l, \; \forall \; l \in \mathcal{L},
\end{equation}
%
where $\mathtt{Nc0}^{\text{LB}}_l$ and $\mathtt{Nc0}^{\text{LB}}_l$ represent
lower and upper bounds of the new capacity variable $\mathtt{Nc0}_{l}$. This is
then linked to the new plant capacity variable $\mathtt{ncap}_{ijl} =
\mathtt{Nc0}_{l}$. Then, similarly as the general and retrofit case, the
variables of the new plant are put and permuted into a column vector for each
period/sub-period and location, $x^n_{ijl}$ (prefix with an $\mathtt{N}$,
e.g.\ $\mathtt{Nu}_{ijl}$).
The constraint is written a follows,
%
\begin{equation}
    \myveebar_{k \in \mathcal{N}}
    \begin{bmatrix}
        Y^n_{ijlk} \\
        A^n_{ijlk} x^n_{ijl} \leq b^n_{ijlk}
    \end{bmatrix}
    , \; \forall i\in \mathcal{P}, j\in \mathcal{P}_1, l\in \mathcal{L}.
\end{equation}
%
with respective matrices and right-hand-side for each new plant technology,
$A^n_{ijlk}$ and $b^n_{ijlk}$. In the same way as retrofits, this constraint
impose a single linear system according to the state of its Boolean variable
$Y^n_{ijlk}$.
%
An important aspect of new plants is that they are assumed to only be
available after an existing or retrofitted plant has been retired. This is, they
must \emph{replace}. To adopt this policy, a logic relationship between the new
plant Boolean variables and the online state of the plant (retrofitted or
otherwise) must be used. Given this, the next section explains the online state
issues.
%
\section{Online state}
%
Plants can be retrofitted, expanded, or remain unchanged. Moreover, they can be
shut down, which removes its associated quantities, i.e.\ the variables of
Eqs.~\eqref{eq:eqset1}--\eqref{eq:eqset2} after the retrofit. 
This action also opens the possibility of replacement by a new plant at a
later time slice. To accommodate this possibility, an additional discrete
decision is required for every time point and location. This essentially takes
the variables from Eq.~\eqref{eq:rf} and nullifies them as follows,
%
\begin{equation}\label{eq:online_Stat}
    \begin{bmatrix}
        Y^o_{ijl} \\
        x^o_{ijl} = x^r_{ijl}
    \end{bmatrix}
    \myveebar
    \begin{bmatrix}
        \neg Y^o_{ijl} \\
        x^o_{ijl} = \bm{0}
    \end{bmatrix}
    , \; \forall i\in \mathcal{P}, j\in \mathcal{P}_1, l\in \mathcal{L}.
\end{equation}
%
In this equation, $Y^o_{ijl} = \left\{\text{True},\text{False}\right\}$
represents the online state of the plant and $x^o_{ijl}$ acts like a conduit for
the enabling the shut down and outputs of the plant. In the online case 
($Y^o_{ijl} = \text{True}$), this vector is set to $x^r_{ijl}$, i.e.\ the
retrofit variable vector.
%
An additional side-effect
of a plant shutdown is the generation of a \emph{retirement cost}, which
requires a single time cost increase equal to the current loan balance, which
can be modelled as follows,
%
\begin{equation}\label{eq:retcost}
    \begin{bmatrix}
        \hat{Y}^o_{ijl} \\
        \mathtt{OretCost}_{ijl} = \mathtt{RloanP}_{ijl}
    \end{bmatrix}
    \myveebar
    \begin{bmatrix}
        \neg \hat{Y}^o_{ijl} \\
        \mathtt{OretCost}_{ijl} = 0
    \end{bmatrix}
    , \; \forall i\in \mathcal{P}, j\in \mathcal{P}_1, l\in \mathcal{L}.
\end{equation}
%
Specific logic relations must be specified between $Y^{o}_{ijl}$ and
$\hat{Y}^o_{ijl}$ to enable the constraints to work properly, as well as
the calculation of the loan balance cost.
%
\section{Logic relations}\label{sec:logic_relations}
%
As mentioned in earlier sections, logic relationships are constraints on the
possible outcomes of the Boolean variables. They encode rules
that provide an additional layer of meaning to the discrete decision of the
model, e.g.\ the way in which a new plant can be created. 
These statements are created using propositional logic with connectives like
implication, equivalence, negation, etc. Ultimately these get transformed
systematically to achieve the \emph{Conjunctive Normal Form} or CNF, which can
be systematically translated into an linear system with binary variables
matching Eq.\eqref{eq:linear_logic}.
%
Thus, the most salient logic relationships are stated in this section. 
%
\paragraph{Expansion}
%
Once the plant is expanded, then the plant remains expanded.
%
\begin{equation}
    Y^e_{ijl} \implies Y^e_{i, j+1, l}, \; \forall 
    i \in \mathcal{P},
    j \in \mathcal{P}_1 \setminus \left\{\mathtt{NsubPer}\right\},
    l\in \mathcal{L}.
\end{equation}
%
\paragraph{Online state}
% once offline
An online plant follows an immediately previous online state.
%
\begin{equation}
    Y^o_{i,j+1,l} \implies Y^o_{ijl}, \; \forall 
    i \in \mathcal{P},
    j \in \mathcal{P}_1 \setminus \left\{\mathtt{NsubPer}\right\},
    l\in \mathcal{L}.
\end{equation}
%
\paragraph{Retrofit}
%
Once the plant is retrofitted, the plant remains retrofitted.
%
\begin{equation}
    Y^r_{ijlk} \implies Y^r_{i, j+1, lk}, \; \forall 
    i \in \mathcal{P},
    j \in \mathcal{P}_1 \setminus \left\{\mathtt{NsubPer}\right\},
    l\in \mathcal{L},
    k\in \mathcal{R} \setminus \left\{\mathtt{1}\right\}.
\end{equation}
%
Non-retrofit state ($r=\mathtt{1}$) follows an immediately previous non-retrofit
state.
%
\begin{equation}
    Y^r_{i, j+1, l,\mathtt{1}} \implies Y^r_{ijl\mathtt{1}} , \; \forall 
    i \in \mathcal{P},
    j \in \mathcal{P}_1 \setminus \left\{\mathtt{NsubPer}\right\},
    l\in \mathcal{L}.
\end{equation}
%
Offline state (\emph{not} online) locks the retrofit state.
%
\begin{equation}
    \neg Y^o_{ijl} \implies \left[Y^r_{ijlk} \iff Y^r_{i,j+1,lk}\right]
    , \; \forall
    i \in \mathcal{P},
    j \in \mathcal{P}_1 \setminus \left\{\mathtt{NsubPer}\right\},
    l \in \mathcal{L},
    k \in \mathcal{K},
\end{equation}
%
\paragraph{New}
%
Once a new plant is created, it remains a new plant.
%
\begin{equation}
    Y^n_{ijlk} \implies Y^n_{i, j+1, lk}, \; \forall 
    i \in \mathcal{P},
    j \in \mathcal{P}_1 \setminus \left\{\mathtt{NsubPer}\right\},
    l\in \mathcal{L},
    k\in \mathcal{N} \setminus \left\{\mathtt{1}\right\}.
\end{equation}
%
Once a new plant is created, it remains a new plant.
%
\begin{equation}
    Y^n_{i, j+1, l,\mathtt{1}} \implies Y^n_{ijl\mathtt{1}} , \; \forall 
    i \in \mathcal{P},
    j \in \mathcal{P}_1 \setminus \left\{\mathtt{NsubPer}\right\},
    l\in \mathcal{L}.
\end{equation}
%
Non-new state ($n=\mathtt{1}$) follows an immediately previous non-new
state, i.e.\ no new plant has been created.
%
\begin{equation}
    Y^o_{ijl} \implies \neg Y^n_{ijlk}, \;
    \forall i \in \mathcal{P} \setminus \left\{1\right\},
    j \in \mathcal{P}_1 \setminus \left\{1\right\},
    l \in \mathcal{L},
    k \in \mathcal{N} \setminus \left\{\mathtt{1}\right\}.
\end{equation}
%
\paragraph{Investment activation (expansion, retrofit, new, etc.)}
%a
These are dummy variables that indicate the time-slice at which transition has
happened, and are expressed here for each possible change of state of the model.
The following equations represent the, retrofit, new-plant, and retirement
statements respectively,
%
\begin{equation}
    \check{Y}^e_{ijl} \implies Y^e_{ijl} \wedge \neg Y^e_{i,j-1,l}, \;
    \forall i \in \mathcal{P} ,
    j \in \mathcal{P}_1 \setminus \left\{1\right\},
    l \in \mathcal{L},
\end{equation}
%
\begin{equation}
    \check{Y}^r_{ijl} \implies 
    \neg Y^r_{ijl,\mathtt{1}} \wedge 
    Y^r_{i,j-1,l,\mathtt{1}}, \;
    \forall i \in \mathcal{P} ,
    j \in \mathcal{P}_1 \setminus \left\{1\right\},
    l \in \mathcal{L},
\end{equation}

\begin{equation}
    \check{Y}^n_{ijl} \implies 
    \neg Y^n_{ijl,\mathtt{1}} \wedge 
    Y^n_{i,j-1,l,\mathtt{1}}, \;
    \forall i \in \mathcal{P} ,
    j \in \mathcal{P}_1 \setminus \left\{1\right\},
    l \in \mathcal{L},
\end{equation}
%
and,
%
\begin{equation}
    \hat{Y}^o_{ijl} \implies 
    \neg Y^o_{ijl} \wedge 
    Y^n_{i,j+1,l}, \;
    \forall i \in \mathcal{P} ,
    j \in \mathcal{P}_1 \setminus \left\{1\right\},
    l \in \mathcal{L}.
\end{equation}

% only offline if retrofit

% once expansion
% expansion/payment status
% retrofit/payment status
% new/payment status
% once newplant
% onff/new
% initial condition
%
\section{Money flow}
%
For expansion, retrofit, and new plants, the capital cost is disaggregated over
time. This is, the overnight capital cost is sliced into a number of periods
(e.g.  30 years), and interest is added at each year. In the model, this is
implemented through a \emph{balance} for which there is a one-time addition of
total capital (loan), and then yearly payments that gradually decrease the
remaining loan and interest from the previous period until the loan is
zeroed-out, and then the payments are switched off. 
%
This is methodology is used for expansion, retrofit and new plants. Here it is
exemplified for the new plant case,
%
\begin{equation}\begin{split}
    \mathtt{Nloan}_{i,j+1,l} =& \mathtt{Nloan}_{ijl} \cdot \left(1+r\right)^n \\
    &- \mathtt{Npay}_{ijl} \cdot \sum^{n-1}_{k=0} \left(1+r\right)^k 
    - \mathtt{Nladd}_{ijl} \cdot \left(1+r\right)^{n-1}, \\
    &\forall i \in \mathcal{P} ,
    j \in \mathcal{P}_1 \setminus \left\{1\right\},
    l \in \mathcal{L},
\end{split}\end{equation}
%
where $n=\mathtt{NsubPer}$ (number of sub-periods), and $r$ represents the
interest rate ($0\leq r\leq 1$). The variables $\mathtt{Nloan}_{ijl}$,
$\mathtt{Npay}_{ijl}$ $\mathtt{Nladd}_{ijl}$, represent the loan balance
(+interest), the payment, and the additional loan (one-time) to the balance. It
can be deduced that as long as both a sufficiently large $\mathtt{Npay}_{ijl}$
value and a zero $\mathtt{Nladd}_{ijl}$ value, the $\mathtt{Nloan}_{ijl}$ will
decrease. 
%
To deactivate the payment variable, the following disaggregation of the loan
balance is created,
%
\begin{equation}
    \mathtt{Nloan}_{ijl} = \mathtt{NloanP}_{ijl} - \mathtt{NloanN}_{ijl}
    , \; \forall i\in \mathcal{P}, j\in \mathcal{P}_1, l\in \mathcal{L},
\end{equation}
%
where $\mathtt{NloanP}_{ijl}, \mathtt{NloanN}_{ijl}\geq 0$ are non-negative. 
To remain feasible, if $\mathtt{Nloan}_{ijl} \ge 0$ then 
$\mathtt{Nloan}_{ijl} = \mathtt{NloanP}_{ijl}$, otherwise 
$\mathtt{Nloan}_{ijl} = \mathtt{NloanN}_{ijl}$. These situations work in
conjunction with the following disjunction to create a \emph{switch} for the
payment,
%
\begin{equation}
    \begin{bmatrix}
        \mathtt{nYps}_{ijl} \\
        \mathtt{NloanP}_{ijl} = 0, \\
        \mathtt{NloanN}_{ijl} \geq 0,\\
        \mathtt{Npay}_{ijl} = 0, \\
    \end{bmatrix}
    \myveebar
    \begin{bmatrix}
        \neg \mathtt{nYps}_{ijl} \\
        \mathtt{NloanP}_{ijl} \geq 0, \\
        \mathtt{NloanN}_{ijl} = 0, \\
        \mathtt{Npay}_{ijl} = \mathtt{NannCost}_{ijl}, \\
    \end{bmatrix}
    , \; \forall i\in \mathcal{P}, j\in \mathcal{P}_1, l\in \mathcal{L}.
\end{equation}
%
where $\mathtt{nYps}_{ijl} \in \left\{\text{True, False}\right\}$ denotes the
\emph{payment state}, i.e.\ fully payed or not yet payed. This disjunction
\emph{deactivates} the payment $\mathtt{Npay}_{ijl}$ when the loan balance
reaches zero, otherwise is set to the annuity calculated from the disaggregated 
overnight capital cost. 
%
At the time-slice in which the new-plant occurs the total capital cost from the
investment is allocated into the $\mathtt{Nladd}_{ijl}$ variable, which is
represented in the following constraint,
%
\begin{equation}
    \begin{bmatrix}
        \check{Y}^n_{ijl} \\
        \mathtt{Nladd}_{ijl} = \mathtt{NovNc}_{ijl}
    \end{bmatrix}
    \myveebar
    \begin{bmatrix}
        \neg \check{Y}^n_{ijl} \\
        \mathtt{Nladd}_{ijl} = 0
    \end{bmatrix}
    , \; \forall i\in \mathcal{P}, j\in \mathcal{P}_1, l\in \mathcal{L}.
\end{equation}
%
This uses the dummy Boolean variable $\check{Y}^n_{ijl}$ described in the
previous section, and is only be able to activate once through the horizon for
each plant, thus achieving the loan increase.

Therefore these last sets of constraints enable the activation and deactivation
of payment costs as well as providing the \emph{remaining} loan balance, which
in the case of existing plants, can be used for the calculation of the
retirement cost according to Eq.\eqref{eq:retcost}.
%
\section{Additional costs}
%
%
On addition to the capital and operating expenditures, the costs of fuel,
electricity and CO2 transport and storage are considered. These are given for
the existing plants (retrofitted or otherwise), and the new plants.
Here the equations for the existing plants are given\footnote{the prefix
$\mathtt{O}$ denotes the online state, such that if the plant is retired these
values are 0.}, and these are fuel, electricity and CO2 respectively.
%
\begin{equation}
    \mathtt{OcFuel}_{ijl} = \sum_{f \in \mathcal{F}} 
    \mathtt{CfuelR}_{ijlf} \mathtt{Ofu}_{ijlf},
    \forall \; i \in \mathcal{P},j \in \mathcal{P}_1 ,
    l \in \mathcal{L},
\end{equation}
%
\begin{equation}
    \mathtt{OcElec}_{ijl} = \mathtt{CelecR}_{ijl} \mathtt{Ou}_{ijl},
    \forall \; i \in \mathcal{P},j \in \mathcal{P}_1 , l \in \mathcal{L},
\end{equation}
%
\begin{equation}
    \mathtt{OcCcus}_{ijl} = \mathtt{CccusR}_{ijl} \mathtt{Oep1gcs}_{ijl},
    \forall \; i \in \mathcal{P},j \in \mathcal{P}_1 , l \in \mathcal{L},
\end{equation}
%
\section{Additional constraints}
% debt financing
The last set of constraints reflects issues related with the period-to-period
adoption of technology, demand, carbon dioxide.  For instance, market shares can
be imposed as upper bounds on specific capacities that result from certain
technologies. This requires a disaggregation of output capacity by retrofit and
new plant technology, and it is given by the variable $\mathtt{OcapRd}_{ijlk}$
for existing plants and $\mathtt{NcapRd}_{ijlk}$ for new plants. The constraint
is then written next,
% Market share
\begin{equation}
    \sum_{l\in\mathcal{L}} 
    \left(\mathtt{OcapRd}_{ijlk} + \mathtt{NcapD}_{ijlk}\right)
    \leq \mathtt{msUB}_{ijk}
    \sum_{l\in\mathcal{L}} 
    \left(\mathtt{Ocap}_{ijl} + \mathtt{Ncap}_{ijl} \right),
    \forall \; i \in \mathcal{P},j \in \mathcal{P}_1 , l \in \mathcal{L}, k \in
    \mathcal{K},
\end{equation}
%
where $\mathtt{Ocap}_{ijl}$ and $\mathtt{Ncap}_{ijl}$ are the existing and new
plant capacities respectively. Furthermore, it requires the specification of the
market share fractions $\mathtt{msUB}_{ijk}\in \left[0, 1\right]$. It is noted
that this constraint also assumes that $\mathcal{R}=\mathcal{N}=\mathcal{K}$.
Next, overall capacity is considered to be bounded by the demand over time,
% demand
\begin{equation}
    \sum_{l\in\mathcal{L}}
    \left(\mathtt{Ocap}_{ijl} + \mathtt{Ncap}_{ijl}\right) 
    \geq \mathtt{demand}_{ij},
    \forall \; i \in \mathcal{P},j \in \mathcal{P}_1 , l \in \mathcal{L}
\end{equation}.
%
Finally, the carbon emission can be bounded in a number of ways. Here a simple
upper bound for every period is considered to bound the un-captured emissions for
existing and new plants $\mathtt{Oep1ge}_{ijl}$ and $\mathtt{Nep1ge}_{ijl}$
respectively, as follows,
% emission
\begin{equation}
    \sum_{l\in\mathcal{L}} 
    \left(\mathtt{Oep1ge}_{ijl} + \mathtt{Nep1ge}_{ijl}\right) \leq 
    \mathtt{CO2Bound}_{ijk},
    \forall \; i \in \mathcal{P},j \in \mathcal{P}_1 , l \in \mathcal{L}.
\end{equation}

%
\section{Objective}
%
The objective function is the minimization of the aggregated discounted cost for
the time horizon for all the plants. It requires specification of a discount
coefficient $\delta_{ij}$, and it has components of capital payment, O\&M, fuel,
electricity and carbon storage and transport for existing (prefix \texttt{O})
and new plants (prefix \texttt{N}), i.e.
%
\begin{equation}
    \begin{split}
        \min \sum_{i\in \mathcal{P}}
        \sum_{j\in \mathcal{P}_1}
        \sum_{l\in \mathcal{L}}
        \delta_{ij}  
        &\left[\mathtt{Opay}_{ijl} 
        + \mathtt{Npay}_{ijl} 
        + \mathtt{OcOnM}_{ijl}
        + \mathtt{NcOnM}_{ijl} \right. \\ 
        &+ \mathtt{OcFuel}_{ijl} 
        + \mathtt{NcFuel}_{ijl} 
        + \mathtt{OcElec}_{ijl} 
        + \mathtt{NcElec}_{ijl}  \\
        &\left.+ \mathtt{OcCcus}_{ijl} 
        + \mathtt{NcCcus}_{ijl}
        + \mathtt{OretCost}_{ijl} 
        + \right].
    \end{split}
\end{equation}

%

%
\appendix
%
\section{Additional Equations}
%
\paragraph{New plant additional costs} These include fuel, electricity and CO2
transport and storage.
\begin{equation}
    \mathtt{NcFuel}_{ijlf} = \mathtt{CFuelR}_{ijlf} \mathtt{Nfu}_{ijlf},
    \forall \; i \in \mathcal{P},j \in \mathcal{P}_1 ,
    l \in \mathcal{L}, f \in \mathcal{F},
\end{equation}
%
\begin{equation}
    \mathtt{NcElec}_{ijl} = \mathtt{CElecR}_{ijl} \mathtt{Nu}_{ijl},
    \forall \; i \in \mathcal{P},j \in \mathcal{P}_1 , l \in \mathcal{L},
\end{equation}
%
\begin{equation}
    \mathtt{NcCcus}_{ijl} = \mathtt{CccusR}_{ijl} \mathtt{Nep1ge}_{ijl},
    \forall \; i \in \mathcal{P},j \in \mathcal{P}_1 , l \in \mathcal{L}.
\end{equation}

%%%%
\paragraph{Expansion and retrofit loan balances}
%
\begin{equation}\begin{split}
    \mathtt{Eloan}_{i,j+1,l} =& \mathtt{Eloan}_{ijl} \cdot \left(1+r\right)^n \\
    &- \mathtt{Epay}_{ijl} \cdot \sum^{n-1}_{k=0} \left(1+r\right)^k 
    - \mathtt{Eladd}_{ijl} \cdot \left(1+r\right)^{n-1}, \\
    &\forall i \in \mathcal{P} ,
    j \in \mathcal{P}_1 \setminus \left\{1\right\},
    l \in \mathcal{L},
\end{split}\end{equation}
%
\begin{equation}
    \mathtt{Eloan}_{ijl} = \mathtt{EloanP}_{ijl} - \mathtt{EloanN}_{ijl}
    , \; \forall i\in \mathcal{P}, j\in \mathcal{P}_1, l\in \mathcal{L}.
\end{equation}
%
\begin{equation}
    \begin{bmatrix}
        \mathtt{eYps}_{ijl} \\
        \mathtt{EloanP}_{ijl} = 0, \\
        \mathtt{EloanN}_{ijl} \geq 0,\\
        \mathtt{Epay}_{ijl} = 0, \\
    \end{bmatrix}
    \myveebar
    \begin{bmatrix}
        \neg \mathtt{eYps}_{ijl} \\
        \mathtt{EloanP}_{ijl} \geq 0, \\
        \mathtt{EloanN}_{ijl} = 0, \\
        \mathtt{Epay}_{ijl} = \mathtt{EannCost}_{ijl}, \\
    \end{bmatrix}
    , \; \forall i\in \mathcal{P}, j\in \mathcal{P}_1, l\in \mathcal{L}.
\end{equation}
%
%%%%
\begin{equation}\begin{split}
    \mathtt{Rloan}_{i,j+1,l} =& \mathtt{Rloan}_{ijl} \cdot \left(1+r\right)^n \\
    &- \mathtt{Rpay}_{ijl} \cdot \sum^{n-1}_{k=0} \left(1+r\right)^k 
    - \mathtt{Rladd}_{ijl} \cdot \left(1+r\right)^{n-1}, \\
    &\forall i \in \mathcal{P} ,
    j \in \mathcal{P}_1 \setminus \left\{1\right\},
    l \in \mathcal{L}.
\end{split}\end{equation}
%
\begin{equation}
    \mathtt{Rloan}_{ijl} = \mathtt{RloanP}_{ijl} - \mathtt{RloanN}_{ijl}
    , \; \forall i\in \mathcal{P}, j\in \mathcal{P}_1, l\in \mathcal{L}.
\end{equation}
%
\begin{equation}
    \begin{bmatrix}
        \mathtt{rYps}_{ijl} \\
        \mathtt{RloanP}_{ijl} = 0, \\
        \mathtt{RloanN}_{ijl} \geq 0,\\
        \mathtt{Rpay}_{ijl} = 0, \\
    \end{bmatrix}
    \myveebar
    \begin{bmatrix}
        \neg \mathtt{rYps}_{ijl} \\
        \mathtt{RloanP}_{ijl} \geq 0, \\
        \mathtt{RloanN}_{ijl} = 0, \\
        \mathtt{Rpay}_{ijl} = \mathtt{RannCost}_{ijl}, \\
    \end{bmatrix}
    , \; \forall i\in \mathcal{P}, j\in \mathcal{P}_1, l\in \mathcal{L}.
\end{equation}
%%%%
\paragraph{Logic (period to period)}
%
These are the same equations as in Sec.~\ref{sec:logic_relations}, however they
link two adjacent periods.
\begin{equation}
    Y^e_{i+1,\mathtt{1},l} \implies Y^e_{i, \mathtt{NsubPer}, l}, \; \forall 
    i \in \mathcal{P} \setminus \left\{\mathtt{Nper}\right\},
    l\in \mathcal{L}.
\end{equation}
%
%
\begin{equation}
    Y^o_{i+1,\mathtt{1},l} \implies Y^o_{i,\mathtt{NsubPer},l}, \; \forall 
    i \in \mathcal{P} \setminus \left\{\mathtt{Nper}\right\},
    l\in \mathcal{L}.
\end{equation}
%
%
\begin{equation}
    Y^r_{i,\mathtt{NsubPer},lk} \implies Y^r_{i+1, \mathtt{1}, lk}, \; \forall 
    i \in \mathcal{P} \setminus \left\{\mathtt{Nper}\right\},
    l\in \mathcal{L},
    k\in \mathcal{R} \setminus \left\{\mathtt{1}\right\}.
\end{equation}
%
%
\begin{equation}
    Y^r_{i+1, \mathtt{1}, l,\mathtt{1}} \implies 
    Y^r_{i,\mathtt{NsubPer},l,\mathtt{1}} , \; \forall 
    i \in \mathcal{P} \setminus \left\{\mathtt{Nper}\right\},
    l\in \mathcal{L}.
\end{equation}
%
%
\begin{equation}
    \neg Y^o_{i,\mathtt{NsubPer},l} \implies 
    \left[Y^r_{i,\mathtt{NsubPer},lk} \iff 
    Y^r_{i+1,\mathtt{1},lk}\right]
    , \; \forall
    i \in \mathcal{P} \setminus \left\{\mathtt{Nper}\right\},
    l \in \mathcal{L},
    k \in \mathcal{K}.
\end{equation}
%
%
\begin{equation}
    Y^n_{i,\mathtt{NsubPer},lk} \implies 
    Y^n_{i+1,\mathtt{1}, lk}, \; \forall 
    i \in \mathcal{P} \setminus \left\{\mathtt{Nper}\right\},
    l\in \mathcal{L},
    k\in \mathcal{N} \setminus \left\{\mathtt{1}\right\}.
\end{equation}
%
%
\begin{equation}
    Y^n_{i+1,\mathtt{1}, l,\mathtt{1}} \implies 
    Y^n_{i,\mathtt{NsubPer},l\mathtt{1}} , \; \forall 
    i \in \mathcal{P} \setminus \left\{\mathtt{Nper}\right\},
    l\in \mathcal{L}.
\end{equation}
%
%
%
\begin{equation}
    \check{Y}^e_{i,\mathtt{1},l} \implies 
    Y^e_{i,\mathtt{1},l} \wedge \neg Y^e_{i-1,\mathtt{NsubPer},l}, \;
    \forall 
    i \in \mathcal{P} \setminus \left\{\mathtt{1}\right\},
    i \in \mathcal{P} ,
    l \in \mathcal{L}.
\end{equation}
%
\begin{equation}
    \check{Y}^r_{i,\mathtt{1},l} \implies 
    \neg Y^r_{i,\mathtt{1},l,\mathtt{1}} \wedge 
    Y^r_{i-1,\mathtt{NsubPer},l,\mathtt{1}}, \;
    \forall 
    i \in \mathcal{P} \setminus \left\{\mathtt{1}\right\},
    l \in \mathcal{L}.
\end{equation}

\begin{equation}
    \check{Y}^n_{i,\mathtt{1},l} \implies 
    \neg Y^n_{i,\mathtt{1},l,\mathtt{1}} \wedge 
    Y^n_{i-1,\mathtt{NsubPer},l,\mathtt{1}}, \;
    \forall 
    i \in \mathcal{P} \setminus \left\{\mathtt{1}\right\},
    l \in \mathcal{L}.
\end{equation}
%
%
\begin{equation}
    \hat{Y}^o_{i,\mathtt{NsubPer},l} \implies 
    \neg Y^o_{i,\mathtt{NsubPer},l} \wedge 
    Y^n_{i+1,\mathtt{1},l}, \;
    \forall 
    i \in \mathcal{P} \setminus \left\{\mathtt{Nper}\right\},
    l \in \mathcal{L}.
\end{equation}


\end{document}
